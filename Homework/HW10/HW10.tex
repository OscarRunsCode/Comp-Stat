\documentclass[12pt]{article}
%%\usepackage[T1]{fontenc}
\usepackage[dvipsnames]{xcolor}
%\usepackage{bigfoot} % to allow verbatim in footnote
\usepackage[numbered,framed]{matlab-prettifier}
\usepackage[letterpaper, margin=1in]{geometry}
\usepackage{subcaption} 

\usepackage[T1]{fontenc}
% Nicer default font (+ math font) than Computer Modern for most use cases
\usepackage{mathpazo}

% Basic figure setup, for now with no caption control since it's done
% automatically by Pandoc (which extracts ![](path) syntax from Markdown).
\usepackage{graphicx}
% We will generate all images so they have a width \maxwidth. This means
% that they will get their normal width if they fit onto the page, but
% are scaled down if they would overflow the margins.
\makeatletter
\def\maxwidth{\ifdim\Gin@nat@width>\linewidth\linewidth
	\else\Gin@nat@width\fi}
\makeatother
\let\Oldincludegraphics\includegraphics
% Set max figure width to be 80% of text width, for now hardcoded.
\renewcommand{\includegraphics}[1]{\Oldincludegraphics[width=.8\maxwidth]{#1}}
% Ensure that by default, figures have no caption (until we provide a
% proper Figure object with a Caption API and a way to capture that
% in the conversion process - todo).
\usepackage{caption}
\DeclareCaptionLabelFormat{nolabel}{}
\captionsetup{labelformat=nolabel}

 \usepackage[breakable]{tcolorbox}
\usepackage{adjustbox} % Used to constrain images to a maximum size 
\usepackage{xcolor} % Allow colors to be defined
\usepackage{enumerate} % Needed for markdown enumerations to work
\usepackage{geometry} % Used to adjust the document margins
\usepackage{amsmath} % Equations
\usepackage{amssymb} % Equations
\usepackage{textcomp} % defines textquotesingle
% Hack from http://tex.stackexchange.com/a/47451/13684:
\AtBeginDocument{%
	\def\PYZsq{\textquotesingle}% Upright quotes in Pygmentized code
}
\usepackage{upquote} % Upright quotes for verbatim code
\usepackage{eurosym} % defines \euro
\usepackage[mathletters]{ucs} % Extended unicode (utf-8) support
\usepackage[utf8x]{inputenc} % Allow utf-8 characters in the tex document
\usepackage{fancyvrb} % verbatim replacement that allows latex
\usepackage{grffile} % extends the file name processing of package graphics 
% to support a larger range 
% The hyperref package gives us a pdf with properly built
% internal navigation ('pdf bookmarks' for the table of contents,
% internal cross-reference links, web links for URLs, etc.)
\usepackage{hyperref}
\usepackage{longtable} % longtable support required by pandoc >1.10
\usepackage{booktabs}  % table support for pandoc > 1.12.2
\usepackage[inline]{enumitem} % IRkernel/repr support (it uses the enumerate* environment)
\usepackage[normalem]{ulem} % ulem is needed to support strikethroughs (\sout)
% normalem makes italics be italics, not underlines
\usepackage{mathrsfs}

\usepackage{amsmath, amssymb, amsthm}
\usepackage{fancyhdr}
\usepackage{mathtools}
\usepackage{tikz}
\usepackage{enumerate}
\usepackage{microtype}
\usepackage[english]{babel}
%\usepackage[utf8]{inputenc}
\usepackage{cancel}
\usepackage{titlesec}
\usepackage{xfrac}
\usepackage{marginnote}
\usepackage{filecontents}
\usepackage{fancyvrb}
\pagestyle{fancy}
\rhead{Oscar Martinez}
\lhead{STA 5106} 					%Insert subject
\chead{HW 10} 					%Insert Title

\newtheorem{theorem}{Theorem}[section]

\newcommand{\Real}{\mathbb{R}}
\newcommand{\Prob}{\mathbb{P}}
\newcommand{\Lagr}{\mathcal{L}}
\newcommand{\LRA}{\Leftrightarrow}
\newcommand{\LA}{\Leftarrow}
\newcommand{\RA}{\Rightarrow}
\newcommand{\ra}{\rightarrow}
\newcommand{\rsa}{\rightsquigarrow} 
\newcommand\Ccancel[2][black]{\renewcommand\CancelColor{\color{#1}}\cancel{#2}}
\newcommand{\at}{a_{t+1}}
\newcommand{\ct}{c_{t+1}}
\DeclareMathOperator{\EX}{\mathbb{E}}% expected value
\DeclareMathOperator{\Var}{\mathbb{V}}% expected value

\renewcommand{\footrulewidth}{0.2pt}
\renewcommand{\qedsymbol}{$\blacksquare$}
\renewcommand{\thesection}{\arabic{section}.}
\renewcommand{\thesubsection}{(\alph{subsection})}
\renewcommand{\thesubsubsection}{\roman{subsubsection}.)}

 % Colors for the hyperref package
\definecolor{urlcolor}{rgb}{0,.145,.698}
\definecolor{linkcolor}{rgb}{.71,0.21,0.01}
\definecolor{citecolor}{rgb}{.12,.54,.11}

% ANSI colors
\definecolor{ansi-black}{HTML}{3E424D}
\definecolor{ansi-black-intense}{HTML}{282C36}
\definecolor{ansi-red}{HTML}{E75C58}
\definecolor{ansi-red-intense}{HTML}{B22B31}
\definecolor{ansi-green}{HTML}{00A250}
\definecolor{ansi-green-intense}{HTML}{007427}
\definecolor{ansi-yellow}{HTML}{DDB62B}
\definecolor{ansi-yellow-intense}{HTML}{B27D12}
\definecolor{ansi-blue}{HTML}{208FFB}
\definecolor{ansi-blue-intense}{HTML}{0065CA}
\definecolor{ansi-magenta}{HTML}{D160C4}
\definecolor{ansi-magenta-intense}{HTML}{A03196}
\definecolor{ansi-cyan}{HTML}{60C6C8}
\definecolor{ansi-cyan-intense}{HTML}{258F8F}
\definecolor{ansi-white}{HTML}{C5C1B4}
\definecolor{ansi-white-intense}{HTML}{A1A6B2}
\definecolor{ansi-default-inverse-fg}{HTML}{FFFFFF}
\definecolor{ansi-default-inverse-bg}{HTML}{000000}

% commands and environments needed by pandoc snippets
% extracted from the output of `pandoc -s`
\providecommand{\tightlist}{%
	\setlength{\itemsep}{0pt}\setlength{\parskip}{0pt}}
\DefineVerbatimEnvironment{Highlighting}{Verbatim}{commandchars=\\\{\}}
% Add ',fontsize=\small' for more characters per line
\newenvironment{Shaded}{}{}
\newcommand{\KeywordTok}[1]{\textcolor[rgb]{0.00,0.44,0.13}{\textbf{{#1}}}}
\newcommand{\DataTypeTok}[1]{\textcolor[rgb]{0.56,0.13,0.00}{{#1}}}
\newcommand{\DecValTok}[1]{\textcolor[rgb]{0.25,0.63,0.44}{{#1}}}
\newcommand{\BaseNTok}[1]{\textcolor[rgb]{0.25,0.63,0.44}{{#1}}}
\newcommand{\FloatTok}[1]{\textcolor[rgb]{0.25,0.63,0.44}{{#1}}}
\newcommand{\CharTok}[1]{\textcolor[rgb]{0.25,0.44,0.63}{{#1}}}
\newcommand{\StringTok}[1]{\textcolor[rgb]{0.25,0.44,0.63}{{#1}}}
\newcommand{\CommentTok}[1]{\textcolor[rgb]{0.38,0.63,0.69}{\textit{{#1}}}}
\newcommand{\OtherTok}[1]{\textcolor[rgb]{0.00,0.44,0.13}{{#1}}}
\newcommand{\AlertTok}[1]{\textcolor[rgb]{1.00,0.00,0.00}{\textbf{{#1}}}}
\newcommand{\FunctionTok}[1]{\textcolor[rgb]{0.02,0.16,0.49}{{#1}}}
\newcommand{\RegionMarkerTok}[1]{{#1}}
\newcommand{\ErrorTok}[1]{\textcolor[rgb]{1.00,0.00,0.00}{\textbf{{#1}}}}
\newcommand{\NormalTok}[1]{{#1}}

% Additional commands for more recent versions of Pandoc
\newcommand{\ConstantTok}[1]{\textcolor[rgb]{0.53,0.00,0.00}{{#1}}}
\newcommand{\SpecialCharTok}[1]{\textcolor[rgb]{0.25,0.44,0.63}{{#1}}}
\newcommand{\VerbatimStringTok}[1]{\textcolor[rgb]{0.25,0.44,0.63}{{#1}}}
\newcommand{\SpecialStringTok}[1]{\textcolor[rgb]{0.73,0.40,0.53}{{#1}}}
\newcommand{\ImportTok}[1]{{#1}}
\newcommand{\DocumentationTok}[1]{\textcolor[rgb]{0.73,0.13,0.13}{\textit{{#1}}}}
\newcommand{\AnnotationTok}[1]{\textcolor[rgb]{0.38,0.63,0.69}{\textbf{\textit{{#1}}}}}
\newcommand{\CommentVarTok}[1]{\textcolor[rgb]{0.38,0.63,0.69}{\textbf{\textit{{#1}}}}}
\newcommand{\VariableTok}[1]{\textcolor[rgb]{0.10,0.09,0.49}{{#1}}}
\newcommand{\ControlFlowTok}[1]{\textcolor[rgb]{0.00,0.44,0.13}{\textbf{{#1}}}}
\newcommand{\OperatorTok}[1]{\textcolor[rgb]{0.40,0.40,0.40}{{#1}}}
\newcommand{\BuiltInTok}[1]{{#1}}
\newcommand{\ExtensionTok}[1]{{#1}}
\newcommand{\PreprocessorTok}[1]{\textcolor[rgb]{0.74,0.48,0.00}{{#1}}}
\newcommand{\AttributeTok}[1]{\textcolor[rgb]{0.49,0.56,0.16}{{#1}}}
\newcommand{\InformationTok}[1]{\textcolor[rgb]{0.38,0.63,0.69}{\textbf{\textit{{#1}}}}}
\newcommand{\WarningTok}[1]{\textcolor[rgb]{0.38,0.63,0.69}{\textbf{\textit{{#1}}}}}


% Define a nice break command that doesn't care if a line doesn't already
% exist.
\def\br{\hspace*{\fill} \\* }
% Math Jax compatibility definitions
\def\gt{>}
\def\lt{<}
\let\Oldtex\TeX
\let\Oldlatex\LaTeX
\renewcommand{\TeX}{\textrm{\Oldtex}}
\renewcommand{\LaTeX}{\textrm{\Oldlatex}}
% Document parameters
% Document title
 % Pygments definitions

\makeatletter
\def\PY@reset{\let\PY@it=\relax \let\PY@bf=\relax%
	\let\PY@ul=\relax \let\PY@tc=\relax%
	\let\PY@bc=\relax \let\PY@ff=\relax}
\def\PY@tok#1{\csname PY@tok@#1\endcsname}
\def\PY@toks#1+{\ifx\relax#1\empty\else%
	\PY@tok{#1}\expandafter\PY@toks\fi}
\def\PY@do#1{\PY@bc{\PY@tc{\PY@ul{%
				\PY@it{\PY@bf{\PY@ff{#1}}}}}}}
\def\PY#1#2{\PY@reset\PY@toks#1+\relax+\PY@do{#2}}

\expandafter\def\csname PY@tok@w\endcsname{\def\PY@tc##1{\textcolor[rgb]{0.73,0.73,0.73}{##1}}}
\expandafter\def\csname PY@tok@c\endcsname{\let\PY@it=\textit\def\PY@tc##1{\textcolor[rgb]{0.25,0.50,0.50}{##1}}}
\expandafter\def\csname PY@tok@cp\endcsname{\def\PY@tc##1{\textcolor[rgb]{0.74,0.48,0.00}{##1}}}
\expandafter\def\csname PY@tok@k\endcsname{\let\PY@bf=\textbf\def\PY@tc##1{\textcolor[rgb]{0.00,0.50,0.00}{##1}}}
\expandafter\def\csname PY@tok@kp\endcsname{\def\PY@tc##1{\textcolor[rgb]{0.00,0.50,0.00}{##1}}}
\expandafter\def\csname PY@tok@kt\endcsname{\def\PY@tc##1{\textcolor[rgb]{0.69,0.00,0.25}{##1}}}
\expandafter\def\csname PY@tok@o\endcsname{\def\PY@tc##1{\textcolor[rgb]{0.40,0.40,0.40}{##1}}}
\expandafter\def\csname PY@tok@ow\endcsname{\let\PY@bf=\textbf\def\PY@tc##1{\textcolor[rgb]{0.67,0.13,1.00}{##1}}}
\expandafter\def\csname PY@tok@nb\endcsname{\def\PY@tc##1{\textcolor[rgb]{0.00,0.50,0.00}{##1}}}
\expandafter\def\csname PY@tok@nf\endcsname{\def\PY@tc##1{\textcolor[rgb]{0.00,0.00,1.00}{##1}}}
\expandafter\def\csname PY@tok@nc\endcsname{\let\PY@bf=\textbf\def\PY@tc##1{\textcolor[rgb]{0.00,0.00,1.00}{##1}}}
\expandafter\def\csname PY@tok@nn\endcsname{\let\PY@bf=\textbf\def\PY@tc##1{\textcolor[rgb]{0.00,0.00,1.00}{##1}}}
\expandafter\def\csname PY@tok@ne\endcsname{\let\PY@bf=\textbf\def\PY@tc##1{\textcolor[rgb]{0.82,0.25,0.23}{##1}}}
\expandafter\def\csname PY@tok@nv\endcsname{\def\PY@tc##1{\textcolor[rgb]{0.10,0.09,0.49}{##1}}}
\expandafter\def\csname PY@tok@no\endcsname{\def\PY@tc##1{\textcolor[rgb]{0.53,0.00,0.00}{##1}}}
\expandafter\def\csname PY@tok@nl\endcsname{\def\PY@tc##1{\textcolor[rgb]{0.63,0.63,0.00}{##1}}}
\expandafter\def\csname PY@tok@ni\endcsname{\let\PY@bf=\textbf\def\PY@tc##1{\textcolor[rgb]{0.60,0.60,0.60}{##1}}}
\expandafter\def\csname PY@tok@na\endcsname{\def\PY@tc##1{\textcolor[rgb]{0.49,0.56,0.16}{##1}}}
\expandafter\def\csname PY@tok@nt\endcsname{\let\PY@bf=\textbf\def\PY@tc##1{\textcolor[rgb]{0.00,0.50,0.00}{##1}}}
\expandafter\def\csname PY@tok@nd\endcsname{\def\PY@tc##1{\textcolor[rgb]{0.67,0.13,1.00}{##1}}}
\expandafter\def\csname PY@tok@s\endcsname{\def\PY@tc##1{\textcolor[rgb]{0.73,0.13,0.13}{##1}}}
\expandafter\def\csname PY@tok@sd\endcsname{\let\PY@it=\textit\def\PY@tc##1{\textcolor[rgb]{0.73,0.13,0.13}{##1}}}
\expandafter\def\csname PY@tok@si\endcsname{\let\PY@bf=\textbf\def\PY@tc##1{\textcolor[rgb]{0.73,0.40,0.53}{##1}}}
\expandafter\def\csname PY@tok@se\endcsname{\let\PY@bf=\textbf\def\PY@tc##1{\textcolor[rgb]{0.73,0.40,0.13}{##1}}}
\expandafter\def\csname PY@tok@sr\endcsname{\def\PY@tc##1{\textcolor[rgb]{0.73,0.40,0.53}{##1}}}
\expandafter\def\csname PY@tok@ss\endcsname{\def\PY@tc##1{\textcolor[rgb]{0.10,0.09,0.49}{##1}}}
\expandafter\def\csname PY@tok@sx\endcsname{\def\PY@tc##1{\textcolor[rgb]{0.00,0.50,0.00}{##1}}}
\expandafter\def\csname PY@tok@m\endcsname{\def\PY@tc##1{\textcolor[rgb]{0.40,0.40,0.40}{##1}}}
\expandafter\def\csname PY@tok@gh\endcsname{\let\PY@bf=\textbf\def\PY@tc##1{\textcolor[rgb]{0.00,0.00,0.50}{##1}}}
\expandafter\def\csname PY@tok@gu\endcsname{\let\PY@bf=\textbf\def\PY@tc##1{\textcolor[rgb]{0.50,0.00,0.50}{##1}}}
\expandafter\def\csname PY@tok@gd\endcsname{\def\PY@tc##1{\textcolor[rgb]{0.63,0.00,0.00}{##1}}}
\expandafter\def\csname PY@tok@gi\endcsname{\def\PY@tc##1{\textcolor[rgb]{0.00,0.63,0.00}{##1}}}
\expandafter\def\csname PY@tok@gr\endcsname{\def\PY@tc##1{\textcolor[rgb]{1.00,0.00,0.00}{##1}}}
\expandafter\def\csname PY@tok@ge\endcsname{\let\PY@it=\textit}
\expandafter\def\csname PY@tok@gs\endcsname{\let\PY@bf=\textbf}
\expandafter\def\csname PY@tok@gp\endcsname{\let\PY@bf=\textbf\def\PY@tc##1{\textcolor[rgb]{0.00,0.00,0.50}{##1}}}
\expandafter\def\csname PY@tok@go\endcsname{\def\PY@tc##1{\textcolor[rgb]{0.53,0.53,0.53}{##1}}}
\expandafter\def\csname PY@tok@gt\endcsname{\def\PY@tc##1{\textcolor[rgb]{0.00,0.27,0.87}{##1}}}
\expandafter\def\csname PY@tok@err\endcsname{\def\PY@bc##1{\setlength{\fboxsep}{0pt}\fcolorbox[rgb]{1.00,0.00,0.00}{1,1,1}{\strut ##1}}}
\expandafter\def\csname PY@tok@kc\endcsname{\let\PY@bf=\textbf\def\PY@tc##1{\textcolor[rgb]{0.00,0.50,0.00}{##1}}}
\expandafter\def\csname PY@tok@kd\endcsname{\let\PY@bf=\textbf\def\PY@tc##1{\textcolor[rgb]{0.00,0.50,0.00}{##1}}}
\expandafter\def\csname PY@tok@kn\endcsname{\let\PY@bf=\textbf\def\PY@tc##1{\textcolor[rgb]{0.00,0.50,0.00}{##1}}}
\expandafter\def\csname PY@tok@kr\endcsname{\let\PY@bf=\textbf\def\PY@tc##1{\textcolor[rgb]{0.00,0.50,0.00}{##1}}}
\expandafter\def\csname PY@tok@bp\endcsname{\def\PY@tc##1{\textcolor[rgb]{0.00,0.50,0.00}{##1}}}
\expandafter\def\csname PY@tok@fm\endcsname{\def\PY@tc##1{\textcolor[rgb]{0.00,0.00,1.00}{##1}}}
\expandafter\def\csname PY@tok@vc\endcsname{\def\PY@tc##1{\textcolor[rgb]{0.10,0.09,0.49}{##1}}}
\expandafter\def\csname PY@tok@vg\endcsname{\def\PY@tc##1{\textcolor[rgb]{0.10,0.09,0.49}{##1}}}
\expandafter\def\csname PY@tok@vi\endcsname{\def\PY@tc##1{\textcolor[rgb]{0.10,0.09,0.49}{##1}}}
\expandafter\def\csname PY@tok@vm\endcsname{\def\PY@tc##1{\textcolor[rgb]{0.10,0.09,0.49}{##1}}}
\expandafter\def\csname PY@tok@sa\endcsname{\def\PY@tc##1{\textcolor[rgb]{0.73,0.13,0.13}{##1}}}
\expandafter\def\csname PY@tok@sb\endcsname{\def\PY@tc##1{\textcolor[rgb]{0.73,0.13,0.13}{##1}}}
\expandafter\def\csname PY@tok@sc\endcsname{\def\PY@tc##1{\textcolor[rgb]{0.73,0.13,0.13}{##1}}}
\expandafter\def\csname PY@tok@dl\endcsname{\def\PY@tc##1{\textcolor[rgb]{0.73,0.13,0.13}{##1}}}
\expandafter\def\csname PY@tok@s2\endcsname{\def\PY@tc##1{\textcolor[rgb]{0.73,0.13,0.13}{##1}}}
\expandafter\def\csname PY@tok@sh\endcsname{\def\PY@tc##1{\textcolor[rgb]{0.73,0.13,0.13}{##1}}}
\expandafter\def\csname PY@tok@s1\endcsname{\def\PY@tc##1{\textcolor[rgb]{0.73,0.13,0.13}{##1}}}
\expandafter\def\csname PY@tok@mb\endcsname{\def\PY@tc##1{\textcolor[rgb]{0.40,0.40,0.40}{##1}}}
\expandafter\def\csname PY@tok@mf\endcsname{\def\PY@tc##1{\textcolor[rgb]{0.40,0.40,0.40}{##1}}}
\expandafter\def\csname PY@tok@mh\endcsname{\def\PY@tc##1{\textcolor[rgb]{0.40,0.40,0.40}{##1}}}
\expandafter\def\csname PY@tok@mi\endcsname{\def\PY@tc##1{\textcolor[rgb]{0.40,0.40,0.40}{##1}}}
\expandafter\def\csname PY@tok@il\endcsname{\def\PY@tc##1{\textcolor[rgb]{0.40,0.40,0.40}{##1}}}
\expandafter\def\csname PY@tok@mo\endcsname{\def\PY@tc##1{\textcolor[rgb]{0.40,0.40,0.40}{##1}}}
\expandafter\def\csname PY@tok@ch\endcsname{\let\PY@it=\textit\def\PY@tc##1{\textcolor[rgb]{0.25,0.50,0.50}{##1}}}
\expandafter\def\csname PY@tok@cm\endcsname{\let\PY@it=\textit\def\PY@tc##1{\textcolor[rgb]{0.25,0.50,0.50}{##1}}}
\expandafter\def\csname PY@tok@cpf\endcsname{\let\PY@it=\textit\def\PY@tc##1{\textcolor[rgb]{0.25,0.50,0.50}{##1}}}
\expandafter\def\csname PY@tok@c1\endcsname{\let\PY@it=\textit\def\PY@tc##1{\textcolor[rgb]{0.25,0.50,0.50}{##1}}}
\expandafter\def\csname PY@tok@cs\endcsname{\let\PY@it=\textit\def\PY@tc##1{\textcolor[rgb]{0.25,0.50,0.50}{##1}}}

\def\PYZbs{\char`\\}
\def\PYZus{\char`\_}
\def\PYZob{\char`\{}
\def\PYZcb{\char`\}}
\def\PYZca{\char`\^}
\def\PYZam{\char`\&}
\def\PYZlt{\char`\<}
\def\PYZgt{\char`\>}
\def\PYZsh{\char`\#}
\def\PYZpc{\char`\%}
\def\PYZdl{\char`\$}
\def\PYZhy{\char`\-}
\def\PYZsq{\char`\'}
\def\PYZdq{\char`\"}
\def\PYZti{\char`\~}
% for compatibility with earlier versions
\def\PYZat{@}
\def\PYZlb{[}
\def\PYZrb{]}
\makeatother


    % For linebreaks inside Verbatim environment from package fancyvrb. 
\makeatletter
\newbox\Wrappedcontinuationbox 
\newbox\Wrappedvisiblespacebox 
\newcommand*\Wrappedvisiblespace {\textcolor{red}{\textvisiblespace}} 
\newcommand*\Wrappedcontinuationsymbol {\textcolor{red}{\llap{\tiny$\m@th\hookrightarrow$}}} 
\newcommand*\Wrappedcontinuationindent {3ex } 
\newcommand*\Wrappedafterbreak {\kern\Wrappedcontinuationindent\copy\Wrappedcontinuationbox} 
% Take advantage of the already applied Pygments mark-up to insert 
% potential linebreaks for TeX processing. 
%        {, <, #, %, $, ' and ": go to next line. 
%        _, }, ^, &, >, - and ~: stay at end of broken line. 
% Use of \textquotesingle for straight quote. 
\newcommand*\Wrappedbreaksatspecials {% 
	\def\PYGZus{\discretionary{\char`\_}{\Wrappedafterbreak}{\char`\_}}% 
	\def\PYGZob{\discretionary{}{\Wrappedafterbreak\char`\{}{\char`\{}}% 
	\def\PYGZcb{\discretionary{\char`\}}{\Wrappedafterbreak}{\char`\}}}% 
	\def\PYGZca{\discretionary{\char`\^}{\Wrappedafterbreak}{\char`\^}}% 
	\def\PYGZam{\discretionary{\char`\&}{\Wrappedafterbreak}{\char`\&}}% 
	\def\PYGZlt{\discretionary{}{\Wrappedafterbreak\char`\<}{\char`\<}}% 
	\def\PYGZgt{\discretionary{\char`\>}{\Wrappedafterbreak}{\char`\>}}% 
	\def\PYGZsh{\discretionary{}{\Wrappedafterbreak\char`\#}{\char`\#}}% 
	\def\PYGZpc{\discretionary{}{\Wrappedafterbreak\char`\%}{\char`\%}}% 
	\def\PYGZdl{\discretionary{}{\Wrappedafterbreak\char`\$}{\char`\$}}% 
	\def\PYGZhy{\discretionary{\char`\-}{\Wrappedafterbreak}{\char`\-}}% 
	\def\PYGZsq{\discretionary{}{\Wrappedafterbreak\textquotesingle}{\textquotesingle}}% 
	\def\PYGZdq{\discretionary{}{\Wrappedafterbreak\char`\"}{\char`\"}}% 
	\def\PYGZti{\discretionary{\char`\~}{\Wrappedafterbreak}{\char`\~}}% 
} 
% Some characters . , ; ? ! / are not pygmentized. 
% This macro makes them "active" and they will insert potential linebreaks 
\newcommand*\Wrappedbreaksatpunct {% 
	\lccode`\~`\.\lowercase{\def~}{\discretionary{\hbox{\char`\.}}{\Wrappedafterbreak}{\hbox{\char`\.}}}% 
	\lccode`\~`\,\lowercase{\def~}{\discretionary{\hbox{\char`\,}}{\Wrappedafterbreak}{\hbox{\char`\,}}}% 
	\lccode`\~`\;\lowercase{\def~}{\discretionary{\hbox{\char`\;}}{\Wrappedafterbreak}{\hbox{\char`\;}}}% 
	\lccode`\~`\:\lowercase{\def~}{\discretionary{\hbox{\char`\:}}{\Wrappedafterbreak}{\hbox{\char`\:}}}% 
	\lccode`\~`\?\lowercase{\def~}{\discretionary{\hbox{\char`\?}}{\Wrappedafterbreak}{\hbox{\char`\?}}}% 
	\lccode`\~`\!\lowercase{\def~}{\discretionary{\hbox{\char`\!}}{\Wrappedafterbreak}{\hbox{\char`\!}}}% 
	\lccode`\~`\/\lowercase{\def~}{\discretionary{\hbox{\char`\/}}{\Wrappedafterbreak}{\hbox{\char`\/}}}% 
	\catcode`\.\active
	\catcode`\,\active 
	\catcode`\;\active
	\catcode`\:\active
	\catcode`\?\active
	\catcode`\!\active
	\catcode`\/\active 
	\lccode`\~`\~ 	
}
\makeatother

\let\OriginalVerbatim=\Verbatim
\makeatletter
\renewcommand{\Verbatim}[1][1]{%
	%\parskip\z@skip
	\sbox\Wrappedcontinuationbox {\Wrappedcontinuationsymbol}%
	\sbox\Wrappedvisiblespacebox {\FV@SetupFont\Wrappedvisiblespace}%
	\def\FancyVerbFormatLine ##1{\hsize\linewidth
		\vtop{\raggedright\hyphenpenalty\z@\exhyphenpenalty\z@
			\doublehyphendemerits\z@\finalhyphendemerits\z@
			\strut ##1\strut}%
	}%
	% If the linebreak is at a space, the latter will be displayed as visible
	% space at end of first line, and a continuation symbol starts next line.
	% Stretch/shrink are however usually zero for typewriter font.
	\def\FV@Space {%
		\nobreak\hskip\z@ plus\fontdimen3\font minus\fontdimen4\font
		\discretionary{\copy\Wrappedvisiblespacebox}{\Wrappedafterbreak}
		{\kern\fontdimen2\font}%
	}%
	
	% Allow breaks at special characters using \PYG... macros.
	\Wrappedbreaksatspecials
	% Breaks at punctuation characters . , ; ? ! and / need catcode=\active 	
	\OriginalVerbatim[#1,codes*=\Wrappedbreaksatpunct]%
}
\makeatother

% Exact colors from NB
\definecolor{incolor}{HTML}{303F9F}
\definecolor{outcolor}{HTML}{D84315}
\definecolor{cellborder}{HTML}{CFCFCF}
\definecolor{cellbackground}{HTML}{F7F7F7}

% prompt
\makeatletter
\newcommand{\boxspacing}{\kern\kvtcb@left@rule\kern\kvtcb@boxsep}
\makeatother
\newcommand{\prompt}[4]{
	\ttfamily\llap{{\color{#2}[#3]:\hspace{3pt}#4}}\vspace{-\baselineskip}
}



% Prevent overflowing lines due to hard-to-break entities
\sloppy 
% Setup hyperref package
\hypersetup{
	breaklinks=true,  % so long urls are correctly broken across lines
	colorlinks=true,
	urlcolor=urlcolor,
	linkcolor=linkcolor,
	citecolor=citecolor,
}
% Slightly bigger margins than the latex defaults



\let\ph\mlplaceholder % shorter macro
\lstMakeShortInline"

\lstset{
	style              = Matlab-editor,
	basicstyle         = \mlttfamily,
	escapechar         = ",
	mlshowsectionrules = true,
}


\begin{document}
	
	 
%	\lstlistoflistings
%	\subsubsection*{Output:}
%	\begin{center}
%		\VerbatimInput{MATLAB_Output_OM.txt}
%	\end{center}

\section*{Problem 1}
	The problem is to use importance sampling to estimate:
	\begin{align*}
		\theta &=\int_{0}^{\infty} x \frac{e^{-(y-x)^{2} / 2} e^{-3 x}}{Z} d x \\
		Z&=\int_{0}^{\infty} e^{-(y-x)^{2} / 2} e^{-3 x} d x;\quad y=0.5
	\end{align*}
	Observe:
	\begin{align*}
	\theta &=\int_{0}^{\infty} x \frac{e^{-(y-x)^{2} / 2} e^{-3 x}}{Z} d x \\
	&=\int_{0}^{\infty} x \frac{e^{-(y-x)^{2} / 2} e^{-3 x}}{\int_{0}^{\infty} e^{-(y-x)^{2} / 2} e^{-3 x} d x} d x\\
	&=\int_{0}^{\infty} x \frac{e^{-(y-x)^{2} / 2}\ 3*e^{-3 x}}{\int_{0}^{\infty} e^{-(y-x)^{2} / 2}\ 3*e^{-3 x} d x} d x
	\end{align*}
	Recall that the pdf, $ f(x) $, of an exponentially distributed random variable, x, with parameter $ \lambda =  \frac{1}{\beta} $ is:
	\[ \beta*e^{-\beta x} \]
	Thus $ 3*e^{-3 x} $ is the pdf of an exponentially distributed r.v. with $ \lambda = \frac{1}{3} $. Our problem thus becomes:
	\begin{align*}
		\theta &=\int_{0}^{\infty} x \frac{e^{-(y-x)^{2} / 2}\ 3*e^{-3 x}}{\int_{0}^{\infty} e^{-(y-x)^{2} / 2}\ 3*e^{-3 x} d x} d x\\
		&=\int_{0}^{\infty}  \frac{x*e^{-(y-x)^{2} / 2}\ 3*e^{-3 x}}{\EX[e^{-(y-x)^{2} / 2}] } d x \\
		\theta &= \frac{\EX[x*e^{-(y-x)^{2} / 2}] }{\EX[e^{-(y-x)^{2} / 2}] }
	\end{align*}
	By the Weak Law of Large Numbers: \[ \lim\limits_{n \to \infty} \Pr\left( \left| \sum\limits_{i=1}^n x_i*e^{-(y-x_i)^{2} / 2} - \EX[x*e^{-(y-x)^{2} / 2}] \right| > \varepsilon \right) = 0 \]
	
	Thus, we can employ the estimator:
		\[ \hat{\theta} = \frac{\sum\limits_{i=1}^n[x_i*e^{-(0.5-x_i)^{2} / 2}] }{\sum\limits_{i=1}^n[e^{-(0.5-x_i)^{2} / 2}] } \]
	Where $ x_i\overset{i.i.d}{\sim}\exp(\frac{1}{3}) $. 
	
	
		\begin{center}
			\adjustimage{max size={0.9\linewidth}{0.9\paperheight}}{Fig1.png}
		\end{center}
	
	
	
\section*{Problem 2}
	The problem is to use tilted sampling to estimate the quantity $ \theta = \Pr(X>a) $ for $ X\sim \mathcal{N}(0,1) $. \\
	Recall that the tilted density is given by:
	\begin{align*}
		f_t(x) &= \frac{\exp(tx)f(x)}{M(t)} \\
		M(t) &= \int_{-\infty}^{\infty} \exp(tx)f(x) d x
	\end{align*}
	Focusing on $ M(t) $, we note that this is simply the moment generating function of a standard normal distribution and is thus:
	\[ M(t) = \exp\left(\frac{t^2}{2}\right) \]
	Thus, the tilted density becomes:
	\begin{align*}
		f_t(x) &= \frac{\exp(tx)f(x)}{M(t)} \\
		&= \frac{\exp(tx)\frac{1}{\sqrt{2 \pi }} \exp\left(-\frac{(x)^{2}}{2 }\right)}{\exp\left(\frac{t^2}{2}\right)}\\
		f_t(x) &= \frac{\exp\left( \frac{-(x-t)^2}{2} \right)}{\sqrt{2 \pi }}
	\end{align*}
	The tilted density is thus normal with mean $ t $ and variance $ 1 $. Choosing $ t $ such that $ a $ is the mean of this tilted density implies that $ t = a $. Now,
	\begin{align*}
		f_t(x) &= \frac{\exp(tx)f(x)}{M(t)} \\
		\LRA \frac{f(x)}{f_t(x)} &= \exp(-ax)M(a) \\
		& = \exp\left(\frac{a^2}{2}-ax\right)\\
		\overset{WLLN}{\RA} \hat{\theta} &= \frac{1}{n}\sum\limits_{i=1}^n I_{x_i\geq a} \exp\left(\frac{a^2}{2}-ax_i\right)
	\end{align*}
		\begin{center}
			\adjustimage{max size={0.9\linewidth}{0.9\paperheight}}{Fig2.png}
		\end{center}
	
\section*{Problem 3}
	The problem is to use tilted sampling to estimate the quantity $ \theta = \Pr(X>a) $ for $ X\sim exp(\frac{1}{\lambda}) $. \\
	Recall that the tilted density is given by:
	\begin{align*}
	f_t(x) &= \frac{\exp(tx)f(x)}{M(t)} \\
	M(t) &= \int_{-\infty}^{\infty} \exp(tx)f(x) d x
	\end{align*}
	Focusing on $ M(t) $:
	\begin{align*}
		M(t) &= \int_{0}^{\infty} \exp(tx) \lambda \exp(-\lambda x) dx\\
		 &= \lambda \int_{0}^{\infty} \exp((t-\lambda)x) dx \\
		 &= \lambda \frac{\exp((t-\lambda)x)}{t-\lambda} \bigg\rvert_{0}^\infty \\ 
		 &= \frac{\lambda}{\lambda-t}\\
	\end{align*}
		Before proceeding further, it is important to note that for all of this to work, it must be the case that $ t < \lambda $.
	\begin{align*}
		 f_t(x) &= \frac{\exp(tx)f(x)}{M(t)} \\
		 &= \frac{\exp(tx)\lambda \exp(-\lambda x))}{\frac{\lambda}{\lambda-t}} \\
		 &= (\lambda-t)\exp(x(t-\lambda))\\
		 &= (\lambda-t)\exp(-(\lambda-t)x)
	\end{align*}
	The tilted density is thus exponentially distributed with corresponding parameter $ \beta = \frac{1}{\lambda - t} $ (i.e. $ exp(\frac{1}{\beta}) $). Recall that the mean of an exponential distribution with parameter $ \frac{1}{\beta} $ is $ \beta $. Choosing the optimal $ t $ to estimate $ \theta $ for a given $ a $ requires $ a $ to be the mean of the tilted density. Thus: \[ \frac{1}{\lambda - t} = a \LRA t = \lambda - \frac{1}{a} \]
	So our distribution is $ exp\left( a \right) $. 
	\begin{align*}
		f_t(x) &= \frac{\exp(tx)f(x)}{M(t)} \\
		\LRA \frac{f(x)}{f_t(x)} &= \exp(-t(a)x)M(t(a))\\
		&=\exp\left(-\left( \lambda - \frac{1}{a} \right)x\right)*\frac{\lambda}{\lambda-\lambda + \frac{1}{a}} \\
		& = \exp\left(-\left( \lambda - \frac{1}{a} \right)x\right)*\lambda a \\
		\overset{WLLN}{\RA} \hat{\theta} &= \frac{1}{n}\sum\limits_{i=1}^n I_{x_i\geq a} \exp\left(-\left( \lambda - \frac{1}{a} \right)x_i\right)*\lambda a
	\end{align*}

		\begin{center}
			\adjustimage{max size={0.9\linewidth}{0.9\paperheight}}{Fig3.png}
		\end{center}
	
\section*{MATLAB Code for Problems 1, 2, 3:}
	\lstinputlisting[]{HW10.m}
	
	
\section*{Problem 4}
   
  
  
  \begin{tcolorbox}[breakable, size=fbox, boxrule=1pt, pad at break*=1mm,colback=cellbackground, colframe=cellborder]
  	\prompt{In}{incolor}{1}{\boxspacing}
  	\begin{Verbatim}[commandchars=\\\{\}]
  	\PY{k+kn}{import} \PY{n+nn}{numpy} \PY{k}{as} \PY{n+nn}{np}
  	\PY{k+kn}{from} \PY{n+nn}{matplotlib} \PY{k}{import} \PY{n}{pyplot}
  	\PY{k+kn}{import} \PY{n+nn}{random}
  	
  	\PY{c+c1}{\PYZsh{}\PYZhy{}\PYZhy{}\PYZhy{}\PYZhy{}\PYZhy{}\PYZhy{}\PYZhy{}\PYZhy{}\PYZhy{}\PYZhy{}\PYZhy{}\PYZhy{}\PYZhy{}\PYZhy{}\PYZhy{}\PYZhy{}\PYZhy{}\PYZhy{}\PYZhy{}\PYZhy{}\PYZhy{}\PYZhy{}\PYZhy{}\PYZhy{}\PYZhy{}\PYZhy{}\PYZhy{}\PYZhy{}\PYZhy{}Problem 4\PYZhy{}\PYZhy{}\PYZhy{}\PYZhy{}\PYZhy{}\PYZhy{}\PYZhy{}\PYZhy{}\PYZhy{}\PYZhy{}\PYZhy{}\PYZhy{}\PYZhy{}\PYZhy{}\PYZhy{}\PYZhy{}\PYZhy{}\PYZhy{}\PYZhy{}\PYZhy{}\PYZhy{}\PYZhy{}\PYZhy{}\PYZhy{}\PYZhy{}\PYZhy{}\PYZhy{}\PYZhy{}\PYZhy{}\PYZhy{}\PYZhy{}\PYZhy{}\PYZhy{}\PYZhy{}}
  	\PY{n+nb}{print}\PY{p}{(}\PY{l+s+s1}{\PYZsq{}}\PY{l+s+s1}{Problem 4}\PY{l+s+s1}{\PYZsq{}}\PY{p}{)}
  	\PY{n}{random}\PY{o}{.}\PY{n}{seed}\PY{p}{(}\PY{l+m+mi}{1}\PY{p}{)}
  	\PY{n}{n} \PY{o}{=} \PY{l+m+mi}{10}\PY{o}{*}\PY{o}{*}\PY{l+m+mi}{7}
  	\PY{n}{K} \PY{o}{=} \PY{l+m+mi}{10}
  	\PY{n}{lm} \PY{o}{=} \PY{l+m+mi}{1}\PY{o}{/}\PY{l+m+mi}{3}
  	\PY{n}{x} \PY{o}{=} \PY{n}{np}\PY{o}{.}\PY{n}{zeros}\PY{p}{(}\PY{p}{(}\PY{n}{n}\PY{p}{,}\PY{n}{K}\PY{p}{)}\PY{p}{)}
  	\PY{n}{theta} \PY{o}{=} \PY{n}{np}\PY{o}{.}\PY{n}{zeros}\PY{p}{(}\PY{p}{(}\PY{l+m+mi}{1}\PY{p}{,}\PY{l+m+mi}{10}\PY{p}{)}\PY{p}{)}
  	\PY{n}{num} \PY{o}{=} \PY{n}{np}\PY{o}{.}\PY{n}{zeros}\PY{p}{(}\PY{p}{(}\PY{l+m+mi}{1}\PY{p}{,}\PY{l+m+mi}{10}\PY{p}{)}\PY{p}{)}
  	\PY{n}{den} \PY{o}{=} \PY{n}{np}\PY{o}{.}\PY{n}{zeros}\PY{p}{(}\PY{p}{(}\PY{l+m+mi}{1}\PY{p}{,}\PY{l+m+mi}{10}\PY{p}{)}\PY{p}{)}
  	
  	
  	\PY{k}{for} \PY{n}{j} \PY{o+ow}{in} \PY{n+nb}{range}\PY{p}{(}\PY{n}{K}\PY{p}{)}\PY{p}{:}
  	\PY{n}{x}\PY{p}{[}\PY{p}{:}\PY{p}{,}\PY{n}{j}\PY{p}{]} \PY{o}{=} \PY{n}{np}\PY{o}{.}\PY{n}{array}\PY{p}{(} \PY{p}{[}\PY{n}{random}\PY{o}{.}\PY{n}{expovariate}\PY{p}{(}\PY{l+m+mi}{1}\PY{o}{/}\PY{n}{lm}\PY{p}{)} \PY{k}{for} \PY{n}{x} \PY{o+ow}{in} \PY{n+nb}{range}\PY{p}{(}\PY{n}{n}\PY{p}{)}\PY{p}{]} \PY{p}{)}
  	\PY{k}{if} \PY{n}{j} \PY{o}{==} \PY{l+m+mi}{0}\PY{p}{:}
  	\PY{n}{num}\PY{p}{[}\PY{l+m+mi}{0}\PY{p}{,}\PY{n}{j}\PY{p}{]} \PY{o}{=}  \PY{n}{np}\PY{o}{.}\PY{n}{mean}\PY{p}{(} \PY{n}{x}\PY{p}{[}\PY{p}{:}\PY{p}{,}\PY{l+m+mi}{0}\PY{p}{]}\PY{o}{*}\PY{n}{np}\PY{o}{.}\PY{n}{exp}\PY{p}{(} \PY{o}{\PYZhy{}}\PY{l+m+mf}{0.5}\PY{o}{*}\PY{p}{(}\PY{l+m+mf}{0.5}\PY{o}{\PYZhy{}}\PY{n}{x}\PY{p}{[}\PY{p}{:}\PY{p}{,}\PY{l+m+mi}{0}\PY{p}{]}\PY{p}{)}\PY{o}{*}\PY{o}{*}\PY{l+m+mi}{2} \PY{p}{)}\PY{o}{/}\PY{l+m+mi}{3} \PY{p}{)}
  	\PY{n}{den}\PY{p}{[}\PY{l+m+mi}{0}\PY{p}{,}\PY{n}{j}\PY{p}{]} \PY{o}{=} \PY{n}{np}\PY{o}{.}\PY{n}{mean}\PY{p}{(} \PY{n}{np}\PY{o}{.}\PY{n}{exp}\PY{p}{(} \PY{o}{\PYZhy{}}\PY{l+m+mf}{0.5}\PY{o}{*}\PY{p}{(}\PY{l+m+mf}{0.5}\PY{o}{\PYZhy{}}\PY{n}{x}\PY{p}{[}\PY{p}{:}\PY{p}{,}\PY{l+m+mi}{0}\PY{p}{]}\PY{p}{)}\PY{o}{*}\PY{o}{*}\PY{l+m+mi}{2} \PY{p}{)}\PY{o}{/}\PY{l+m+mi}{3} \PY{p}{)}
  	\PY{n}{theta}\PY{p}{[}\PY{l+m+mi}{0}\PY{p}{,}\PY{n}{j}\PY{p}{]} \PY{o}{=} \PY{n}{num}\PY{p}{[}\PY{l+m+mi}{0}\PY{p}{,}\PY{n}{j}\PY{p}{]}\PY{o}{/}\PY{n}{den}\PY{p}{[}\PY{l+m+mi}{0}\PY{p}{,}\PY{n}{j}\PY{p}{]}
  	\PY{k}{else}\PY{p}{:}
  	\PY{n}{num}\PY{p}{[}\PY{l+m+mi}{0}\PY{p}{,}\PY{l+m+mi}{0}\PY{p}{:}\PY{n}{j}\PY{o}{+}\PY{l+m+mi}{1}\PY{p}{]} \PY{o}{=}  \PY{n}{np}\PY{o}{.}\PY{n}{mean}\PY{p}{(} \PY{n}{x}\PY{p}{[}\PY{p}{:}\PY{p}{,}\PY{l+m+mi}{0}\PY{p}{:}\PY{n}{j}\PY{o}{+}\PY{l+m+mi}{1}\PY{p}{]}\PY{o}{*}\PY{n}{np}\PY{o}{.}\PY{n}{exp}\PY{p}{(} \PY{o}{\PYZhy{}}\PY{l+m+mf}{0.5}\PY{o}{*}\PY{p}{(}\PY{l+m+mf}{0.5}\PY{o}{\PYZhy{}}\PY{n}{x}\PY{p}{[}\PY{p}{:}\PY{p}{,}\PY{l+m+mi}{0}\PY{p}{:}\PY{n}{j}\PY{o}{+}\PY{l+m+mi}{1}\PY{p}{]}\PY{p}{)}\PY{o}{*}\PY{o}{*}\PY{l+m+mi}{2} \PY{p}{)} \PY{p}{)}
  	\PY{n}{den}\PY{p}{[}\PY{l+m+mi}{0}\PY{p}{,}\PY{n}{j}\PY{p}{]} \PY{o}{=} \PY{n}{np}\PY{o}{.}\PY{n}{mean}\PY{p}{(} \PY{n}{np}\PY{o}{.}\PY{n}{exp}\PY{p}{(} \PY{o}{\PYZhy{}}\PY{l+m+mf}{0.5}\PY{o}{*}\PY{p}{(}\PY{l+m+mf}{0.5}\PY{o}{\PYZhy{}}\PY{n}{x}\PY{p}{[}\PY{p}{:}\PY{p}{,}\PY{l+m+mi}{0}\PY{p}{:}\PY{n}{j}\PY{o}{+}\PY{l+m+mi}{1}\PY{p}{]}\PY{p}{)}\PY{o}{*}\PY{o}{*}\PY{l+m+mi}{2} \PY{p}{)} \PY{p}{)}
  	\PY{n}{theta}\PY{p}{[}\PY{l+m+mi}{0}\PY{p}{,}\PY{n}{j}\PY{p}{]} \PY{o}{=} \PY{n}{num}\PY{p}{[}\PY{l+m+mi}{0}\PY{p}{,}\PY{n}{j}\PY{p}{]}\PY{o}{/}\PY{n}{den}\PY{p}{[}\PY{l+m+mi}{0}\PY{p}{,}\PY{n}{j}\PY{p}{]}
  	
  	\PY{n}{t} \PY{o}{=} \PY{n+nb}{range}\PY{p}{(}\PY{l+m+mi}{0}\PY{p}{,}\PY{n}{K}\PY{p}{)}
  	\PY{c+c1}{\PYZsh{}Plot}
  	\PY{n}{pyplot}\PY{o}{.}\PY{n}{plot}\PY{p}{(}\PY{n}{t}\PY{p}{,} \PY{n}{theta}\PY{p}{[}\PY{l+m+mi}{0}\PY{p}{,}\PY{p}{:}\PY{p}{]}\PY{p}{,} \PY{l+s+s1}{\PYZsq{}}\PY{l+s+s1}{b}\PY{l+s+s1}{\PYZsq{}}\PY{p}{)}\PY{p}{;}
  	\PY{n}{pyplot}\PY{o}{.}\PY{n}{title}\PY{p}{(}\PY{l+s+s1}{\PYZsq{}}\PY{l+s+s1}{Estimate of Theta}\PY{l+s+s1}{\PYZsq{}}\PY{p}{)}\PY{p}{;}
  	\PY{n}{pyplot}\PY{o}{.}\PY{n}{subplots\PYZus{}adjust}\PY{p}{(}\PY{n}{hspace}\PY{o}{=}\PY{l+m+mi}{1}\PY{p}{,}\PY{n}{wspace}\PY{o}{=}\PY{l+m+mf}{0.5}\PY{p}{)}\PY{p}{;}
  	\PY{n}{pyplot}\PY{o}{.}\PY{n}{figure}\PY{p}{(}\PY{n}{num}\PY{o}{=}\PY{l+m+mi}{1}\PY{p}{,} \PY{n}{figsize}\PY{o}{=}\PY{p}{(}\PY{l+m+mi}{10}\PY{p}{,} \PY{l+m+mi}{10}\PY{p}{)}\PY{p}{,} \PY{n}{dpi}\PY{o}{=}\PY{l+m+mi}{140}\PY{p}{,} \PY{n}{facecolor}\PY{o}{=}\PY{l+s+s1}{\PYZsq{}}\PY{l+s+s1}{w}\PY{l+s+s1}{\PYZsq{}}\PY{p}{,} \PY{n}{edgecolor}\PY{o}{=}\PY{l+s+s1}{\PYZsq{}}\PY{l+s+s1}{k}\PY{l+s+s1}{\PYZsq{}}\PY{p}{)}\PY{p}{;}
  	\PY{n}{pyplot}\PY{o}{.}\PY{n}{xlabel}\PY{p}{(}\PY{l+s+s1}{\PYZsq{}}\PY{l+s+s1}{Iteration K (Sample Size for each iteration is K*10\PYZca{}7)}\PY{l+s+s1}{\PYZsq{}}\PY{p}{)}\PY{p}{;}
  	\PY{n}{pyplot}\PY{o}{.}\PY{n}{ylabel}\PY{p}{(}\PY{l+s+s1}{\PYZsq{}}\PY{l+s+s1}{Estimate of Theta}\PY{l+s+s1}{\PYZsq{}}\PY{p}{)}\PY{p}{;}
  	\PY{n}{pyplot}\PY{o}{.}\PY{n}{show}\PY{p}{(}\PY{p}{)}\PY{p}{;}
  	\end{Verbatim}
  \end{tcolorbox}
  
  \begin{Verbatim}[commandchars=\\\{\}]
  Problem 4
  \end{Verbatim}
  
  \begin{center}
  	\adjustimage{max size={0.9\linewidth}{0.9\paperheight}}{output_0_1.png}
  \end{center}
  { \hspace*{\fill} \\}
  
   \section*{Problem 5}
  \begin{tcolorbox}[breakable, size=fbox, boxrule=1pt, pad at break*=1mm,colback=cellbackground, colframe=cellborder]
  	\prompt{In}{incolor}{2}{\boxspacing}
  	\begin{Verbatim}[commandchars=\\\{\}]
  	\PY{k+kn}{import} \PY{n+nn}{numpy} \PY{k}{as} \PY{n+nn}{np}
  	\PY{k+kn}{from} \PY{n+nn}{matplotlib} \PY{k}{import} \PY{n}{pyplot}
  	\PY{k+kn}{import} \PY{n+nn}{random}
  	
  	\PY{c+c1}{\PYZsh{}\PYZhy{}\PYZhy{}\PYZhy{}\PYZhy{}\PYZhy{}\PYZhy{}\PYZhy{}\PYZhy{}\PYZhy{}\PYZhy{}\PYZhy{}\PYZhy{}\PYZhy{}\PYZhy{}\PYZhy{}\PYZhy{}\PYZhy{}\PYZhy{}\PYZhy{}\PYZhy{}\PYZhy{}\PYZhy{}\PYZhy{}\PYZhy{}\PYZhy{}\PYZhy{}\PYZhy{}\PYZhy{}\PYZhy{}Problem 5\PYZhy{}\PYZhy{}\PYZhy{}\PYZhy{}\PYZhy{}\PYZhy{}\PYZhy{}\PYZhy{}\PYZhy{}\PYZhy{}\PYZhy{}\PYZhy{}\PYZhy{}\PYZhy{}\PYZhy{}\PYZhy{}\PYZhy{}\PYZhy{}\PYZhy{}\PYZhy{}\PYZhy{}\PYZhy{}\PYZhy{}\PYZhy{}\PYZhy{}\PYZhy{}\PYZhy{}\PYZhy{}\PYZhy{}\PYZhy{}\PYZhy{}\PYZhy{}\PYZhy{}\PYZhy{}}
  	\PY{n+nb}{print}\PY{p}{(}\PY{l+s+s1}{\PYZsq{}}\PY{l+s+s1}{Problem 5}\PY{l+s+s1}{\PYZsq{}}\PY{p}{)}
  	\PY{n}{random}\PY{o}{.}\PY{n}{seed}\PY{p}{(}\PY{l+m+mi}{1}\PY{p}{)}
  	\PY{n}{K} \PY{o}{=} \PY{l+m+mi}{5}
  	\PY{n}{i} \PY{o}{=} \PY{l+m+mi}{0}
  	\PY{n}{mu} \PY{o}{=} \PY{n}{np}\PY{o}{.}\PY{n}{zeros}\PY{p}{(}\PY{p}{(}\PY{l+m+mi}{3}\PY{p}{,}\PY{l+m+mi}{5}\PY{p}{)}\PY{p}{)}
  	
  	\PY{k}{for} \PY{n}{y} \PY{o+ow}{in} \PY{n+nb}{range}\PY{p}{(}\PY{l+m+mi}{3}\PY{p}{,}\PY{l+m+mi}{8}\PY{p}{,}\PY{l+m+mi}{2}\PY{p}{)}\PY{p}{:}
  	\PY{c+c1}{\PYZsh{}print(y)}
  	\PY{n}{n} \PY{o}{=} \PY{l+m+mi}{10}\PY{o}{*}\PY{o}{*}\PY{n}{y}
  	\PY{k}{if} \PY{n}{i} \PY{o}{==} \PY{l+m+mi}{3}\PY{p}{:}
  	\PY{n}{i} \PY{o}{=} \PY{l+m+mi}{0}
  	\PY{k}{for} \PY{n}{a} \PY{o+ow}{in} \PY{n+nb}{range}\PY{p}{(}\PY{n}{K}\PY{p}{)}\PY{p}{:}
  	\PY{c+c1}{\PYZsh{}print(a)}
  	\PY{n}{s} \PY{o}{=} \PY{n}{a}\PY{o}{+}\PY{l+m+mi}{1}
  	\PY{n}{x} \PY{o}{=} \PY{n}{np}\PY{o}{.}\PY{n}{array}\PY{p}{(} \PY{p}{[}\PY{n}{random}\PY{o}{.}\PY{n}{normalvariate}\PY{p}{(}\PY{n}{a}\PY{o}{+}\PY{l+m+mi}{1}\PY{p}{,}\PY{l+m+mi}{1}\PY{p}{)} \PY{k}{for} \PY{n}{x} \PY{o+ow}{in} \PY{n+nb}{range}\PY{p}{(}\PY{n}{n}\PY{p}{)}\PY{p}{]} \PY{p}{)}
  	\PY{n}{mu}\PY{p}{[}\PY{n}{i}\PY{p}{,}\PY{n}{a}\PY{p}{]} \PY{o}{=} \PY{n}{np}\PY{o}{.}\PY{n}{mean}\PY{p}{(} \PY{p}{(}\PY{n}{x} \PY{o}{\PYZgt{}} \PY{n}{s}\PY{p}{)}\PY{o}{*}\PY{n}{np}\PY{o}{.}\PY{n}{exp}\PY{p}{(}\PY{l+m+mf}{0.5}\PY{o}{*}\PY{n}{s}\PY{o}{*}\PY{o}{*}\PY{l+m+mi}{2} \PY{o}{\PYZhy{}} \PY{n}{s}\PY{o}{*}\PY{n}{x} \PY{p}{)} \PY{p}{)}
  	\PY{n}{i} \PY{o}{=} \PY{n}{i}\PY{o}{+}\PY{l+m+mi}{1}
  	
  	\PY{k}{for} \PY{n}{a} \PY{o+ow}{in} \PY{n+nb}{range}\PY{p}{(}\PY{n}{K}\PY{p}{)}\PY{p}{:}
  	\PY{c+c1}{\PYZsh{}Plot}
  	\PY{n}{t} \PY{o}{=} \PY{n+nb}{range}\PY{p}{(}\PY{l+m+mi}{0}\PY{p}{,}\PY{l+m+mi}{3}\PY{p}{)}
  	\PY{n}{s} \PY{o}{=} \PY{n}{a}\PY{o}{+}\PY{l+m+mi}{1}
  	\PY{n}{pyplot}\PY{o}{.}\PY{n}{plot}\PY{p}{(}\PY{n}{t}\PY{p}{,} \PY{n}{mu}\PY{p}{[}\PY{p}{:}\PY{p}{,}\PY{n}{a}\PY{p}{]}\PY{p}{,} \PY{l+s+s1}{\PYZsq{}}\PY{l+s+s1}{o\PYZhy{}}\PY{l+s+s1}{\PYZsq{}}\PY{p}{)}\PY{p}{;}
  	\PY{n}{pyplot}\PY{o}{.}\PY{n}{title}\PY{p}{(}\PY{l+s+s1}{\PYZsq{}}\PY{l+s+s1}{a = }\PY{l+s+si}{\PYZpc{}i}\PY{l+s+s1}{\PYZsq{}} \PY{o}{\PYZpc{}}\PY{k}{s});
  	\PY{n}{pyplot}\PY{o}{.}\PY{n}{xlabel}\PY{p}{(}\PY{l+s+s1}{\PYZsq{}}\PY{l+s+s1}{log\PYZus{}10(n)}\PY{l+s+s1}{\PYZsq{}}\PY{p}{)}\PY{p}{;}
  	\PY{n}{pyplot}\PY{o}{.}\PY{n}{show}\PY{p}{(}\PY{p}{)}\PY{p}{;}
  	\end{Verbatim}
  \end{tcolorbox}

  \begin{Verbatim}[commandchars=\\\{\}]
  Problem 5
  \end{Verbatim}
  
  \begin{center}
  	\adjustimage{max size={0.45\linewidth}{0.45\paperheight}}{output_1_1.png}
  	\adjustimage{max size={0.45\linewidth}{0.45\paperheight}}{output_1_2.png}
  \end{center}

	\begin{center}
		\adjustimage{max size={0.45\linewidth}{0.45\paperheight}}{output_1_3.png}
		\adjustimage{max size={0.45\linewidth}{0.45\paperheight}}{output_1_4.png}
	\end{center}
  
  \begin{center}
  	\adjustimage{max size={0.9\linewidth}{0.9\paperheight}}{output_1_5.png}
  \end{center}
  { \hspace*{\fill} \\}




 
	
\end{document}
