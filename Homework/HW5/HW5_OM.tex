\documentclass[12pt]{article}
%%\usepackage[T1]{fontenc}
\usepackage[dvipsnames]{xcolor}
%\usepackage{bigfoot} % to allow verbatim in footnote
\usepackage[numbered,framed]{matlab-prettifier}
\usepackage[letterpaper, margin=1in]{geometry}
\usepackage{subcaption} 

\usepackage[T1]{fontenc}
% Nicer default font (+ math font) than Computer Modern for most use cases
\usepackage{mathpazo}

% Basic figure setup, for now with no caption control since it's done
% automatically by Pandoc (which extracts ![](path) syntax from Markdown).
\usepackage{graphicx}
% We will generate all images so they have a width \maxwidth. This means
% that they will get their normal width if they fit onto the page, but
% are scaled down if they would overflow the margins.
\makeatletter
\def\maxwidth{\ifdim\Gin@nat@width>\linewidth\linewidth
	\else\Gin@nat@width\fi}
\makeatother
\let\Oldincludegraphics\includegraphics
% Set max figure width to be 80% of text width, for now hardcoded.
\renewcommand{\includegraphics}[1]{\Oldincludegraphics[width=.8\maxwidth]{#1}}
% Ensure that by default, figures have no caption (until we provide a
% proper Figure object with a Caption API and a way to capture that
% in the conversion process - todo).
\usepackage{caption}
\DeclareCaptionLabelFormat{nolabel}{}
\captionsetup{labelformat=nolabel}

\usepackage{adjustbox} % Used to constrain images to a maximum size 
\usepackage{xcolor} % Allow colors to be defined
\usepackage{enumerate} % Needed for markdown enumerations to work
\usepackage{geometry} % Used to adjust the document margins
\usepackage{amsmath} % Equations
\usepackage{amssymb} % Equations
\usepackage{textcomp} % defines textquotesingle
% Hack from http://tex.stackexchange.com/a/47451/13684:
\AtBeginDocument{%
	\def\PYZsq{\textquotesingle}% Upright quotes in Pygmentized code
}
\usepackage{upquote} % Upright quotes for verbatim code
\usepackage{eurosym} % defines \euro
\usepackage[mathletters]{ucs} % Extended unicode (utf-8) support
\usepackage[utf8x]{inputenc} % Allow utf-8 characters in the tex document
\usepackage{fancyvrb} % verbatim replacement that allows latex
\usepackage{grffile} % extends the file name processing of package graphics 
% to support a larger range 
% The hyperref package gives us a pdf with properly built
% internal navigation ('pdf bookmarks' for the table of contents,
% internal cross-reference links, web links for URLs, etc.)
\usepackage{hyperref}
\usepackage{longtable} % longtable support required by pandoc >1.10
\usepackage{booktabs}  % table support for pandoc > 1.12.2
\usepackage[inline]{enumitem} % IRkernel/repr support (it uses the enumerate* environment)
\usepackage[normalem]{ulem} % ulem is needed to support strikethroughs (\sout)
% normalem makes italics be italics, not underlines
\usepackage{mathrsfs}

\usepackage{amsmath, amssymb, amsthm}
\usepackage{fancyhdr}
\usepackage{mathtools}
\usepackage{tikz}
\usepackage{enumerate}
\usepackage{microtype}
\usepackage[english]{babel}
%\usepackage[utf8]{inputenc}
\usepackage{cancel}
\usepackage{titlesec}
\usepackage{xfrac}
\usepackage{marginnote}
%\usepackage [autostyle, english = american]{csquotes}
%\MakeOuterQuote{"}
\usepackage{filecontents}
\usepackage{fancyvrb}



\pagestyle{fancy}
\rhead{Oscar Martinez}
\lhead{STA 5106} 					%Insert subject
\chead{HW 5} 					%Insert Title

\newtheorem{theorem}{Theorem}[section]

\newcommand{\Real}{\mathbb{R}}
\newcommand{\Prob}{\mathbb{P}}
\newcommand{\Lagr}{\mathcal{L}}
\newcommand{\LRA}{\Leftrightarrow}
\newcommand{\LA}{\Leftarrow}
\newcommand{\RA}{\Rightarrow}
\newcommand{\ra}{\rightarrow}
\newcommand{\rsa}{\rightsquigarrow} 
\newcommand\Ccancel[2][black]{\renewcommand\CancelColor{\color{#1}}\cancel{#2}}
\newcommand{\at}{a_{t+1}}
\newcommand{\ct}{c_{t+1}}
\DeclareMathOperator{\EX}{\mathbb{E}}% expected value
\DeclareMathOperator{\Var}{\mathbb{V}}% expected value

\renewcommand{\footrulewidth}{0.2pt}
\renewcommand{\qedsymbol}{$\blacksquare$}
\renewcommand{\thesection}{\arabic{section}.}
\renewcommand{\thesubsection}{(\alph{subsection})}
\renewcommand{\thesubsubsection}{\roman{subsubsection}.)}

 % Colors for the hyperref package
\definecolor{urlcolor}{rgb}{0,.145,.698}
\definecolor{linkcolor}{rgb}{.71,0.21,0.01}
\definecolor{citecolor}{rgb}{.12,.54,.11}

% ANSI colors
\definecolor{ansi-black}{HTML}{3E424D}
\definecolor{ansi-black-intense}{HTML}{282C36}
\definecolor{ansi-red}{HTML}{E75C58}
\definecolor{ansi-red-intense}{HTML}{B22B31}
\definecolor{ansi-green}{HTML}{00A250}
\definecolor{ansi-green-intense}{HTML}{007427}
\definecolor{ansi-yellow}{HTML}{DDB62B}
\definecolor{ansi-yellow-intense}{HTML}{B27D12}
\definecolor{ansi-blue}{HTML}{208FFB}
\definecolor{ansi-blue-intense}{HTML}{0065CA}
\definecolor{ansi-magenta}{HTML}{D160C4}
\definecolor{ansi-magenta-intense}{HTML}{A03196}
\definecolor{ansi-cyan}{HTML}{60C6C8}
\definecolor{ansi-cyan-intense}{HTML}{258F8F}
\definecolor{ansi-white}{HTML}{C5C1B4}
\definecolor{ansi-white-intense}{HTML}{A1A6B2}
\definecolor{ansi-default-inverse-fg}{HTML}{FFFFFF}
\definecolor{ansi-default-inverse-bg}{HTML}{000000}

% commands and environments needed by pandoc snippets
% extracted from the output of `pandoc -s`
\providecommand{\tightlist}{%
	\setlength{\itemsep}{0pt}\setlength{\parskip}{0pt}}
\DefineVerbatimEnvironment{Highlighting}{Verbatim}{commandchars=\\\{\}}
% Add ',fontsize=\small' for more characters per line
\newenvironment{Shaded}{}{}
\newcommand{\KeywordTok}[1]{\textcolor[rgb]{0.00,0.44,0.13}{\textbf{{#1}}}}
\newcommand{\DataTypeTok}[1]{\textcolor[rgb]{0.56,0.13,0.00}{{#1}}}
\newcommand{\DecValTok}[1]{\textcolor[rgb]{0.25,0.63,0.44}{{#1}}}
\newcommand{\BaseNTok}[1]{\textcolor[rgb]{0.25,0.63,0.44}{{#1}}}
\newcommand{\FloatTok}[1]{\textcolor[rgb]{0.25,0.63,0.44}{{#1}}}
\newcommand{\CharTok}[1]{\textcolor[rgb]{0.25,0.44,0.63}{{#1}}}
\newcommand{\StringTok}[1]{\textcolor[rgb]{0.25,0.44,0.63}{{#1}}}
\newcommand{\CommentTok}[1]{\textcolor[rgb]{0.38,0.63,0.69}{\textit{{#1}}}}
\newcommand{\OtherTok}[1]{\textcolor[rgb]{0.00,0.44,0.13}{{#1}}}
\newcommand{\AlertTok}[1]{\textcolor[rgb]{1.00,0.00,0.00}{\textbf{{#1}}}}
\newcommand{\FunctionTok}[1]{\textcolor[rgb]{0.02,0.16,0.49}{{#1}}}
\newcommand{\RegionMarkerTok}[1]{{#1}}
\newcommand{\ErrorTok}[1]{\textcolor[rgb]{1.00,0.00,0.00}{\textbf{{#1}}}}
\newcommand{\NormalTok}[1]{{#1}}

% Additional commands for more recent versions of Pandoc
\newcommand{\ConstantTok}[1]{\textcolor[rgb]{0.53,0.00,0.00}{{#1}}}
\newcommand{\SpecialCharTok}[1]{\textcolor[rgb]{0.25,0.44,0.63}{{#1}}}
\newcommand{\VerbatimStringTok}[1]{\textcolor[rgb]{0.25,0.44,0.63}{{#1}}}
\newcommand{\SpecialStringTok}[1]{\textcolor[rgb]{0.73,0.40,0.53}{{#1}}}
\newcommand{\ImportTok}[1]{{#1}}
\newcommand{\DocumentationTok}[1]{\textcolor[rgb]{0.73,0.13,0.13}{\textit{{#1}}}}
\newcommand{\AnnotationTok}[1]{\textcolor[rgb]{0.38,0.63,0.69}{\textbf{\textit{{#1}}}}}
\newcommand{\CommentVarTok}[1]{\textcolor[rgb]{0.38,0.63,0.69}{\textbf{\textit{{#1}}}}}
\newcommand{\VariableTok}[1]{\textcolor[rgb]{0.10,0.09,0.49}{{#1}}}
\newcommand{\ControlFlowTok}[1]{\textcolor[rgb]{0.00,0.44,0.13}{\textbf{{#1}}}}
\newcommand{\OperatorTok}[1]{\textcolor[rgb]{0.40,0.40,0.40}{{#1}}}
\newcommand{\BuiltInTok}[1]{{#1}}
\newcommand{\ExtensionTok}[1]{{#1}}
\newcommand{\PreprocessorTok}[1]{\textcolor[rgb]{0.74,0.48,0.00}{{#1}}}
\newcommand{\AttributeTok}[1]{\textcolor[rgb]{0.49,0.56,0.16}{{#1}}}
\newcommand{\InformationTok}[1]{\textcolor[rgb]{0.38,0.63,0.69}{\textbf{\textit{{#1}}}}}
\newcommand{\WarningTok}[1]{\textcolor[rgb]{0.38,0.63,0.69}{\textbf{\textit{{#1}}}}}


% Define a nice break command that doesn't care if a line doesn't already
% exist.
\def\br{\hspace*{\fill} \\* }
% Math Jax compatibility definitions
\def\gt{>}
\def\lt{<}
\let\Oldtex\TeX
\let\Oldlatex\LaTeX
\renewcommand{\TeX}{\textrm{\Oldtex}}
\renewcommand{\LaTeX}{\textrm{\Oldlatex}}
% Document parameters
% Document title
 % Pygments definitions

\makeatletter
\def\PY@reset{\let\PY@it=\relax \let\PY@bf=\relax%
	\let\PY@ul=\relax \let\PY@tc=\relax%
	\let\PY@bc=\relax \let\PY@ff=\relax}
\def\PY@tok#1{\csname PY@tok@#1\endcsname}
\def\PY@toks#1+{\ifx\relax#1\empty\else%
	\PY@tok{#1}\expandafter\PY@toks\fi}
\def\PY@do#1{\PY@bc{\PY@tc{\PY@ul{%
				\PY@it{\PY@bf{\PY@ff{#1}}}}}}}
\def\PY#1#2{\PY@reset\PY@toks#1+\relax+\PY@do{#2}}

\expandafter\def\csname PY@tok@w\endcsname{\def\PY@tc##1{\textcolor[rgb]{0.73,0.73,0.73}{##1}}}
\expandafter\def\csname PY@tok@c\endcsname{\let\PY@it=\textit\def\PY@tc##1{\textcolor[rgb]{0.25,0.50,0.50}{##1}}}
\expandafter\def\csname PY@tok@cp\endcsname{\def\PY@tc##1{\textcolor[rgb]{0.74,0.48,0.00}{##1}}}
\expandafter\def\csname PY@tok@k\endcsname{\let\PY@bf=\textbf\def\PY@tc##1{\textcolor[rgb]{0.00,0.50,0.00}{##1}}}
\expandafter\def\csname PY@tok@kp\endcsname{\def\PY@tc##1{\textcolor[rgb]{0.00,0.50,0.00}{##1}}}
\expandafter\def\csname PY@tok@kt\endcsname{\def\PY@tc##1{\textcolor[rgb]{0.69,0.00,0.25}{##1}}}
\expandafter\def\csname PY@tok@o\endcsname{\def\PY@tc##1{\textcolor[rgb]{0.40,0.40,0.40}{##1}}}
\expandafter\def\csname PY@tok@ow\endcsname{\let\PY@bf=\textbf\def\PY@tc##1{\textcolor[rgb]{0.67,0.13,1.00}{##1}}}
\expandafter\def\csname PY@tok@nb\endcsname{\def\PY@tc##1{\textcolor[rgb]{0.00,0.50,0.00}{##1}}}
\expandafter\def\csname PY@tok@nf\endcsname{\def\PY@tc##1{\textcolor[rgb]{0.00,0.00,1.00}{##1}}}
\expandafter\def\csname PY@tok@nc\endcsname{\let\PY@bf=\textbf\def\PY@tc##1{\textcolor[rgb]{0.00,0.00,1.00}{##1}}}
\expandafter\def\csname PY@tok@nn\endcsname{\let\PY@bf=\textbf\def\PY@tc##1{\textcolor[rgb]{0.00,0.00,1.00}{##1}}}
\expandafter\def\csname PY@tok@ne\endcsname{\let\PY@bf=\textbf\def\PY@tc##1{\textcolor[rgb]{0.82,0.25,0.23}{##1}}}
\expandafter\def\csname PY@tok@nv\endcsname{\def\PY@tc##1{\textcolor[rgb]{0.10,0.09,0.49}{##1}}}
\expandafter\def\csname PY@tok@no\endcsname{\def\PY@tc##1{\textcolor[rgb]{0.53,0.00,0.00}{##1}}}
\expandafter\def\csname PY@tok@nl\endcsname{\def\PY@tc##1{\textcolor[rgb]{0.63,0.63,0.00}{##1}}}
\expandafter\def\csname PY@tok@ni\endcsname{\let\PY@bf=\textbf\def\PY@tc##1{\textcolor[rgb]{0.60,0.60,0.60}{##1}}}
\expandafter\def\csname PY@tok@na\endcsname{\def\PY@tc##1{\textcolor[rgb]{0.49,0.56,0.16}{##1}}}
\expandafter\def\csname PY@tok@nt\endcsname{\let\PY@bf=\textbf\def\PY@tc##1{\textcolor[rgb]{0.00,0.50,0.00}{##1}}}
\expandafter\def\csname PY@tok@nd\endcsname{\def\PY@tc##1{\textcolor[rgb]{0.67,0.13,1.00}{##1}}}
\expandafter\def\csname PY@tok@s\endcsname{\def\PY@tc##1{\textcolor[rgb]{0.73,0.13,0.13}{##1}}}
\expandafter\def\csname PY@tok@sd\endcsname{\let\PY@it=\textit\def\PY@tc##1{\textcolor[rgb]{0.73,0.13,0.13}{##1}}}
\expandafter\def\csname PY@tok@si\endcsname{\let\PY@bf=\textbf\def\PY@tc##1{\textcolor[rgb]{0.73,0.40,0.53}{##1}}}
\expandafter\def\csname PY@tok@se\endcsname{\let\PY@bf=\textbf\def\PY@tc##1{\textcolor[rgb]{0.73,0.40,0.13}{##1}}}
\expandafter\def\csname PY@tok@sr\endcsname{\def\PY@tc##1{\textcolor[rgb]{0.73,0.40,0.53}{##1}}}
\expandafter\def\csname PY@tok@ss\endcsname{\def\PY@tc##1{\textcolor[rgb]{0.10,0.09,0.49}{##1}}}
\expandafter\def\csname PY@tok@sx\endcsname{\def\PY@tc##1{\textcolor[rgb]{0.00,0.50,0.00}{##1}}}
\expandafter\def\csname PY@tok@m\endcsname{\def\PY@tc##1{\textcolor[rgb]{0.40,0.40,0.40}{##1}}}
\expandafter\def\csname PY@tok@gh\endcsname{\let\PY@bf=\textbf\def\PY@tc##1{\textcolor[rgb]{0.00,0.00,0.50}{##1}}}
\expandafter\def\csname PY@tok@gu\endcsname{\let\PY@bf=\textbf\def\PY@tc##1{\textcolor[rgb]{0.50,0.00,0.50}{##1}}}
\expandafter\def\csname PY@tok@gd\endcsname{\def\PY@tc##1{\textcolor[rgb]{0.63,0.00,0.00}{##1}}}
\expandafter\def\csname PY@tok@gi\endcsname{\def\PY@tc##1{\textcolor[rgb]{0.00,0.63,0.00}{##1}}}
\expandafter\def\csname PY@tok@gr\endcsname{\def\PY@tc##1{\textcolor[rgb]{1.00,0.00,0.00}{##1}}}
\expandafter\def\csname PY@tok@ge\endcsname{\let\PY@it=\textit}
\expandafter\def\csname PY@tok@gs\endcsname{\let\PY@bf=\textbf}
\expandafter\def\csname PY@tok@gp\endcsname{\let\PY@bf=\textbf\def\PY@tc##1{\textcolor[rgb]{0.00,0.00,0.50}{##1}}}
\expandafter\def\csname PY@tok@go\endcsname{\def\PY@tc##1{\textcolor[rgb]{0.53,0.53,0.53}{##1}}}
\expandafter\def\csname PY@tok@gt\endcsname{\def\PY@tc##1{\textcolor[rgb]{0.00,0.27,0.87}{##1}}}
\expandafter\def\csname PY@tok@err\endcsname{\def\PY@bc##1{\setlength{\fboxsep}{0pt}\fcolorbox[rgb]{1.00,0.00,0.00}{1,1,1}{\strut ##1}}}
\expandafter\def\csname PY@tok@kc\endcsname{\let\PY@bf=\textbf\def\PY@tc##1{\textcolor[rgb]{0.00,0.50,0.00}{##1}}}
\expandafter\def\csname PY@tok@kd\endcsname{\let\PY@bf=\textbf\def\PY@tc##1{\textcolor[rgb]{0.00,0.50,0.00}{##1}}}
\expandafter\def\csname PY@tok@kn\endcsname{\let\PY@bf=\textbf\def\PY@tc##1{\textcolor[rgb]{0.00,0.50,0.00}{##1}}}
\expandafter\def\csname PY@tok@kr\endcsname{\let\PY@bf=\textbf\def\PY@tc##1{\textcolor[rgb]{0.00,0.50,0.00}{##1}}}
\expandafter\def\csname PY@tok@bp\endcsname{\def\PY@tc##1{\textcolor[rgb]{0.00,0.50,0.00}{##1}}}
\expandafter\def\csname PY@tok@fm\endcsname{\def\PY@tc##1{\textcolor[rgb]{0.00,0.00,1.00}{##1}}}
\expandafter\def\csname PY@tok@vc\endcsname{\def\PY@tc##1{\textcolor[rgb]{0.10,0.09,0.49}{##1}}}
\expandafter\def\csname PY@tok@vg\endcsname{\def\PY@tc##1{\textcolor[rgb]{0.10,0.09,0.49}{##1}}}
\expandafter\def\csname PY@tok@vi\endcsname{\def\PY@tc##1{\textcolor[rgb]{0.10,0.09,0.49}{##1}}}
\expandafter\def\csname PY@tok@vm\endcsname{\def\PY@tc##1{\textcolor[rgb]{0.10,0.09,0.49}{##1}}}
\expandafter\def\csname PY@tok@sa\endcsname{\def\PY@tc##1{\textcolor[rgb]{0.73,0.13,0.13}{##1}}}
\expandafter\def\csname PY@tok@sb\endcsname{\def\PY@tc##1{\textcolor[rgb]{0.73,0.13,0.13}{##1}}}
\expandafter\def\csname PY@tok@sc\endcsname{\def\PY@tc##1{\textcolor[rgb]{0.73,0.13,0.13}{##1}}}
\expandafter\def\csname PY@tok@dl\endcsname{\def\PY@tc##1{\textcolor[rgb]{0.73,0.13,0.13}{##1}}}
\expandafter\def\csname PY@tok@s2\endcsname{\def\PY@tc##1{\textcolor[rgb]{0.73,0.13,0.13}{##1}}}
\expandafter\def\csname PY@tok@sh\endcsname{\def\PY@tc##1{\textcolor[rgb]{0.73,0.13,0.13}{##1}}}
\expandafter\def\csname PY@tok@s1\endcsname{\def\PY@tc##1{\textcolor[rgb]{0.73,0.13,0.13}{##1}}}
\expandafter\def\csname PY@tok@mb\endcsname{\def\PY@tc##1{\textcolor[rgb]{0.40,0.40,0.40}{##1}}}
\expandafter\def\csname PY@tok@mf\endcsname{\def\PY@tc##1{\textcolor[rgb]{0.40,0.40,0.40}{##1}}}
\expandafter\def\csname PY@tok@mh\endcsname{\def\PY@tc##1{\textcolor[rgb]{0.40,0.40,0.40}{##1}}}
\expandafter\def\csname PY@tok@mi\endcsname{\def\PY@tc##1{\textcolor[rgb]{0.40,0.40,0.40}{##1}}}
\expandafter\def\csname PY@tok@il\endcsname{\def\PY@tc##1{\textcolor[rgb]{0.40,0.40,0.40}{##1}}}
\expandafter\def\csname PY@tok@mo\endcsname{\def\PY@tc##1{\textcolor[rgb]{0.40,0.40,0.40}{##1}}}
\expandafter\def\csname PY@tok@ch\endcsname{\let\PY@it=\textit\def\PY@tc##1{\textcolor[rgb]{0.25,0.50,0.50}{##1}}}
\expandafter\def\csname PY@tok@cm\endcsname{\let\PY@it=\textit\def\PY@tc##1{\textcolor[rgb]{0.25,0.50,0.50}{##1}}}
\expandafter\def\csname PY@tok@cpf\endcsname{\let\PY@it=\textit\def\PY@tc##1{\textcolor[rgb]{0.25,0.50,0.50}{##1}}}
\expandafter\def\csname PY@tok@c1\endcsname{\let\PY@it=\textit\def\PY@tc##1{\textcolor[rgb]{0.25,0.50,0.50}{##1}}}
\expandafter\def\csname PY@tok@cs\endcsname{\let\PY@it=\textit\def\PY@tc##1{\textcolor[rgb]{0.25,0.50,0.50}{##1}}}

\def\PYZbs{\char`\\}
\def\PYZus{\char`\_}
\def\PYZob{\char`\{}
\def\PYZcb{\char`\}}
\def\PYZca{\char`\^}
\def\PYZam{\char`\&}
\def\PYZlt{\char`\<}
\def\PYZgt{\char`\>}
\def\PYZsh{\char`\#}
\def\PYZpc{\char`\%}
\def\PYZdl{\char`\$}
\def\PYZhy{\char`\-}
\def\PYZsq{\char`\'}
\def\PYZdq{\char`\"}
\def\PYZti{\char`\~}
% for compatibility with earlier versions
\def\PYZat{@}
\def\PYZlb{[}
\def\PYZrb{]}
\makeatother


% Exact colors from NB
\definecolor{incolor}{rgb}{0.0, 0.0, 0.5}
\definecolor{outcolor}{rgb}{0.545, 0.0, 0.0}




% Prevent overflowing lines due to hard-to-break entities
\sloppy 
% Setup hyperref package
\hypersetup{
	breaklinks=true,  % so long urls are correctly broken across lines
	colorlinks=true,
	urlcolor=urlcolor,
	linkcolor=linkcolor,
	citecolor=citecolor,
}


\begin{filecontents*}{HW5.m}
	clc
	clear
	
	% Diary
	dfile ='MATLAB_Output_OM.txt';
	if exist(dfile, 'file') ; delete(dfile); end
	diary(dfile)
	diary on
	diary MATLAB_Output_OM.txt
	
	%Introduction
	fprintf('--------------------------------------------------------------\n');
	fprintf('Oscar Martinez \t Homework 5: Problems 1-3 \t STA 5106\n');
	fprintf('--------------------------------------------------------------\n');
	
	%-----Problem 1:-----
	fprintf('-----------Problem 1----------\n');
	
	%Part i
	fprintf('---Part (i): Simple Iterations---\n');
	
	%Define the function inline
	f = inline('0.9*sin(x)-x', 'x');
	
	x0 = pi/4;
	
	xa(1) = x0;
	gxa = f(xa(1))+xa(1);
	i = 1;
	while (abs(xa(i) - gxa) > 1e-6)
	gxa = xa(i);
	xa(i+1) = f(xa(i)) + xa(i);
	i = i+1;
	end
	%xa(i)
	i
	
	
	%Part ii
	fprintf('---Part (ii): Newton-Raphson---\n');
	
	% define the function
	h = inline('0.9*sin(x)-x', 'x');
	x0 = pi/4;
	dh = inline('0.9*cos(x)-1', 'x');
	
	%The N-R algorithm
	clear x;
	x0 = pi/4;
	x(1) = x0;
	gx = x(1)-h(x(1))/dh(x(1));
	i = 1;
	while (abs(x(i) - gx) > 1e-6)
	gx = x(i);
	x(i+1) = x(i) - h(x(i))/dh(x(i));
	i = i+1;
	end
	x
	i
	
	%Part iii
	fprintf('---Part (iii): Plot---\n');
	
	% plot the function
	t = -1:0.1:2;
	yt = f(t);
	figure(1);
	subplot(2,1,1);
	Y=plot(t,yt, 'c-');
	hold on;
	XP=[pi/4 xa(106) x(5)];
	YP=[f(pi/4) f(xa(11)) f(x(5))];
	labels = {'x0','I','N-R'};
	plot(XP,YP,'o');
	text(XP,YP,labels,'VerticalAlignment','bottom','HorizontalAlignment','left')
	Z=plot([-1 1], [0 0], 'r');
	grid on;
	grid minor;
	title('Problem 1')
	axis([-.4 1 -.5 .5]);
	set(gca, 'fontsize', 16);
	legend([Y Z],{'0.9*sin(x)-x','f(x)=0'})
	subplot(2,1,2);
	plot(x,'r');
	grid on;
	grid minor;
	hold on;
	plot(xa,'b');
	set(gca, 'fontsize', 16);
	axis([0 100 -.25 1])
	legend('Newton-Raphson','Iteration')
	hold off;
	
	
	%-----Problem 2:-----
	fprintf('----------Problem 2----------\n');
	clear x;
	
	%Define the function
	f2 = inline('x.^5 - 4.5*x.^4 +4.55*x.^3 + 2.675*x.^2 -3.3*x -1.4375', 'x');
	
	%Find the Roots
	p=[1 -4.5 4.55 2.675 -3.3 -1.4375];
	r=roots(p);
	
	%Find the multiplicity of root "rt"
	rt=-0.5; %Define the wanted root
	[s,t]=size(r);
	m=0;
	for j = 1:s
	if abs(rt-r(j)) < 1e-6
	m=m+1;
	end
	end
	m
	K=(m-1)/m
	
	%Begin the N-R Algorigthm
	x0 = -0.6;
	df2 = inline('5*x.^4 -4*4.5*x.^3 +3*4.55*x.^2 + 2*2.675*x -3.3', 'x');
	clear x;
	x(1) = x0;
	gx = x(1)-f2(x(1))/df2(x(1));
	i = 1;
	while (abs(x(i) - gx) > 1e-6)
	gx = x(i);
	x(i+1) = x(i) - f2(x(i))/df2(x(i));
	i = i+1;
	end
	x
	i
	
	%Alternate way to find K
	%Get E
	for j = 1:i-1
	E(j)=abs(x(j)-x(j+1));
	end
	%K~Ei+1/E
	for j = 1:i-2
	Kalt=min(E(j+1)/E(j));
	end
	Kalt %K=(m-1)/m
	Malt=1/(1-K) %1/(1-K)
	
	%Plot
	figure(2)
	t=[-5:0.1:5];
	yt=f2(t);
	Y=plot(t, yt);
	hold on;
	Z=plot([-5 5], [0 0], 'r');
	%plot(r, 0, 'go', r, f2(r), 'co', x(i), f2(x(i)), 'ro');
	XP2=[-0.6 x(i)];
	YP2=[f2(-0.6) f2(x(18))];
	labels = {'x0','N-R'};
	plot(XP2,YP2,'o');
	text(XP2,YP2,labels,'VerticalAlignment','bottom','HorizontalAlignment','right')
	axis([-2 2 -1 1]);
	legend([Y Z],{'5*x.^4 -4*4.5*x.^3 +3*4.55*x.^2 + 2*2.675*x -3.3','f(x)=0'});
	ax=gca;
	ax.FontSize=16;
	grid on;
	grid minor;
	title('Problem 2')
	hold off;
	
	
	%-----Problem 3:-----
	fprintf('----------Problem 3----------\n');
	
	clear
	%Load the data
	load hw5_3_data.mat
	[m,n]=size(X);
	
	theta0 = 7;
	
	
	%N-R Algo
	theta(1) = theta0;
	%LogL=LogL + x(i) - X(j)-2*log(1+exp(x(i)-X(j)));
	i = 1;
	dtheta = theta(1)+.5;
	while (abs(theta(i) - dtheta) > 1e-6)
	dtheta = theta(i);
	dLogL = sum(1-2*(exp(theta(i)-X)./(1+exp(theta(i)-X))));
	ddLogL = -2*sum(exp(theta(i)-X)./(1+exp(theta(i)-X)).^2);
	theta(i+1) = theta(i) - dLogL/ddLogL;
	i = i+1;
	end
	i
	theta(i)
	
	diary off
\end{filecontents*}

\let\ph\mlplaceholder % shorter macro
\lstMakeShortInline"

\lstset{
	style              = Matlab-editor,
	basicstyle         = \mlttfamily,
	escapechar         = ",
	mlshowsectionrules = true,
}


\begin{document}
	
	%\lstlistoflistings
	\section*{MATLAB (Problems 1-3)}
	\subsection*{Output:}
	\begin{Verbatim}[fontsize=\small]
	--------------------------------------------------------------
	Oscar Martinez 	 Homework 5: Problems 1-3 	 STA 5106
	--------------------------------------------------------------
	-----------Problem 1----------
	---Part (i): Simple Iterations---
	
	i =
	
	106
	
	---Part (ii): Newton-Raphson---
	
	x =
	
	0.7854    0.3756    0.0963    0.0026    0.0000    0.0000
	
	
	i =
	
	6
	
	---Part (iii): Plot---
	----------Problem 2----------
	
	m =
	
	2
	
	
	K =
	
	0.5000
	
	
	x =
	
	Columns 1 through 9
	
	-0.6000   -0.5530   -0.5274   -0.5139   -0.5070   -0.5035   -0.5018   -0.5009   -0.5004
	
	Columns 10 through 18
	
	-0.5002   -0.5001   -0.5001   -0.5000   -0.5000   -0.5000   -0.5000   -0.5000   -0.5000
	
	
	i =
	
	18
	
	
	Kalt =
	
	0.5000
	
	
	Malt =
	
	2
	
	----------Problem 3----------
	
	i =
	
	7
	
	
	ans =
	
	10.0670
	\end{Verbatim}
	
	\subsection*{Figures:}
	\begin{center}
		\adjustimage{max size={0.45\linewidth}{0.45\paperheight}}{P1.png}
		%\end{center}
		%{ \hspace*{\fill} \\}
		%\begin{center}
		\adjustimage{max size={0.45\linewidth}{0.45\paperheight}}{P2.png}
	\end{center}
	
	

	
\subsection*{Code:}
	\lstinputlisting[]{HW5.m}
	
\section*{Problem 3}
	Let $ X_1,\ X_2,\ \dots,\ X_n $ be independent and identically distributed samples from a logistic distribution with the probability density function 
	\[ f(x|\theta)=\frac{\exp(\theta-x)}{(1+\exp(\theta-x))^2} \]
	Given the values of $ X_1,\ X_2,\ \dots,\ X_n $ in ``hw5\_3\_data'' from the blackboard website, our goal is to find the maximum likelihood estimate (MLE) of $ \theta $, using the following steps:
	
	\subsection{}
	Derive an expression for the log likelihood function 
	\[ l(\theta)=\sum_{i=1}^{n}\log(f(X_i|\theta)), \]
	such that the MLE is given by
	\[ \hat{\theta}=\arg\max_{\theta} l(\theta). \]
	\begin{align*}
		l(\theta)&=\sum_{i=1}^{n}\log(f(X_i|\theta)) \\
		&= \sum_{i=1}^{n}\log\left( \frac{\exp(\theta-X_i)}{(1+\exp(\theta-X_i))^2} \right)\\
		&=\sum_{i=1}^{n}\log(\exp(\theta-X_i))-2\log(1+\exp(\theta-X_i))\\
		&=n\theta -\sum_{i=1}^{n}\left( X_i- 2\log(1+\exp(\theta-X_i)) \right)
	\end{align*}	
	
	\subsection{}
	Find the expression for $ \dot{l}(\theta) $ and $ \ddot{l}(\theta) $, the first and the second derivatives of $ l $ with respect to $ \theta $. Verify that $ \ddot{l}(\theta)<0 $.
	\begin{equation*}
	\dot{l}=\frac{d l(\theta)}{d \theta}=n-2\left[\sum_{i=1}^{n} \frac{\exp(\theta-X_i)}{1+\exp(\theta-X_i)} \right]
	\end{equation*}
	\begin{equation*}
	\ddot{l}=\frac{d^2 l(\theta)}{d \theta^2}=-2\left[\sum_{i=1}^{n} \frac{\exp(\theta-X_i)}{(1+\exp(\theta-X_i))^2} \right]
	\end{equation*}
	As $ \exp(\alpha)>0,\quad \forall \alpha\in\mathbb{R} $, the numerator and denominator of this expression are both positive. Thus since the quotient is positive and is being multiplied by a negative number, the expression must be less than 0, as needed.
	
	\subsection{}
	See above output.
	
\section*{Problems 4-5}
 \begin{Verbatim}[commandchars=\\\{\}]
{\color{incolor}In [{\color{incolor}62}]:} \PY{c+c1}{\PYZsh{}Problem 4}
\PY{n+nb}{print}\PY{p}{(}\PY{l+s+s2}{\PYZdq{}}\PY{l+s+s2}{\PYZhy{}\PYZhy{}\PYZhy{}\PYZhy{}\PYZhy{}\PYZhy{}\PYZhy{}Problem 4\PYZhy{}\PYZhy{}\PYZhy{}\PYZhy{}\PYZhy{}\PYZhy{}\PYZhy{}}\PY{l+s+s2}{\PYZdq{}}\PY{p}{)}
\PY{k+kn}{from} \PY{n+nn}{numpy} \PY{k}{import} \PY{o}{*}
\PY{k+kn}{from} \PY{n+nn}{matplotlib} \PY{k}{import} \PY{n}{pyplot}
\PY{n}{set\PYZus{}printoptions}\PY{p}{(}\PY{n}{precision}\PY{o}{=}\PY{l+m+mi}{4}\PY{p}{)}

\PY{c+c1}{\PYZsh{} define the function}
\PY{n}{f} \PY{o}{=} \PY{k}{lambda} \PY{n}{x}\PY{p}{:} \PY{l+m+mf}{0.9}\PY{o}{*}\PY{n}{sin}\PY{p}{(}\PY{n}{x}\PY{p}{)}\PY{o}{\PYZhy{}}\PY{n}{x}
\PY{n}{df} \PY{o}{=} \PY{k}{lambda} \PY{n}{x}\PY{p}{:} \PY{l+m+mf}{0.9}\PY{o}{*}\PY{n}{cos}\PY{p}{(}\PY{n}{x}\PY{p}{)}\PY{o}{\PYZhy{}}\PY{l+m+mi}{1}

\PY{c+c1}{\PYZsh{}Part i}
\PY{n+nb}{print}\PY{p}{(}\PY{l+s+s2}{\PYZdq{}}\PY{l+s+s2}{\PYZhy{}\PYZhy{}\PYZhy{}Part(i) Simple Iterations\PYZhy{}\PYZhy{}\PYZhy{}}\PY{l+s+s2}{\PYZdq{}}\PY{p}{)}
\PY{n}{x0} \PY{o}{=} \PY{n}{pi}\PY{o}{/}\PY{l+m+mi}{4}
\PY{n}{xa0} \PY{o}{=} \PY{n}{pi}\PY{o}{/}\PY{l+m+mi}{4}

\PY{c+c1}{\PYZsh{}Algorithm}
\PY{n}{xa} \PY{o}{=} \PY{n}{zeros}\PY{p}{(}\PY{l+m+mi}{200}\PY{p}{)}
\PY{n}{xa}\PY{p}{[}\PY{l+m+mi}{0}\PY{p}{]}\PY{o}{=}  \PY{n}{xa0}
\PY{n}{gxa} \PY{o}{=} \PY{n}{f}\PY{p}{(}\PY{n}{xa}\PY{p}{[}\PY{l+m+mi}{0}\PY{p}{]}\PY{p}{)} \PY{o}{+} \PY{n}{xa}\PY{p}{[}\PY{l+m+mi}{0}\PY{p}{]}
\PY{n}{i} \PY{o}{=} \PY{l+m+mi}{0}

\PY{k}{while} \PY{n+nb}{abs}\PY{p}{(}\PY{n}{xa}\PY{p}{[}\PY{n}{i}\PY{p}{]} \PY{o}{\PYZhy{}} \PY{n}{gxa}\PY{p}{)} \PY{o}{\PYZgt{}} \PY{l+m+mf}{1e\PYZhy{}6}\PY{p}{:}
\PY{n}{gxa} \PY{o}{=} \PY{n}{xa}\PY{p}{[}\PY{n}{i}\PY{p}{]}
\PY{n}{xa}\PY{p}{[}\PY{n}{i}\PY{o}{+}\PY{l+m+mi}{1}\PY{p}{]} \PY{o}{=} \PY{n}{f}\PY{p}{(}\PY{n}{xa}\PY{p}{[}\PY{n}{i}\PY{p}{]}\PY{p}{)} \PY{o}{+} \PY{n}{xa}\PY{p}{[}\PY{n}{i}\PY{p}{]}
\PY{n}{i} \PY{o}{=} \PY{n}{i} \PY{o}{+} \PY{l+m+mi}{1}

\PY{n}{ia}\PY{o}{=}\PY{n}{i}\PY{o}{+}\PY{l+m+mi}{1}
\PY{n+nb}{print}\PY{p}{(}\PY{l+s+s1}{\PYZsq{}}\PY{l+s+s1}{Iterations until convergence: }\PY{l+s+s1}{\PYZsq{}}\PY{p}{,}\PY{n}{i}\PY{o}{+}\PY{l+m+mi}{1}\PY{p}{)}

\PY{c+c1}{\PYZsh{}Part ii}
\PY{n+nb}{print}\PY{p}{(}\PY{l+s+s2}{\PYZdq{}}\PY{l+s+s2}{\PYZhy{}\PYZhy{}\PYZhy{}Part(ii) Newton\PYZhy{}Raphson\PYZhy{}\PYZhy{}\PYZhy{}}\PY{l+s+s2}{\PYZdq{}}\PY{p}{)}
\PY{c+c1}{\PYZsh{} Newton\PYZhy{}Raphson}
\PY{n}{x0} \PY{o}{=} \PY{n}{pi}\PY{o}{/}\PY{l+m+mi}{4}

\PY{n}{x} \PY{o}{=} \PY{n}{zeros}\PY{p}{(}\PY{l+m+mi}{100}\PY{p}{)}
\PY{n}{x}\PY{p}{[}\PY{l+m+mi}{0}\PY{p}{]}\PY{o}{=}  \PY{n}{x0}
\PY{n}{gx} \PY{o}{=} \PY{n}{x}\PY{p}{[}\PY{l+m+mi}{0}\PY{p}{]} \PY{o}{\PYZhy{}} \PY{n}{f}\PY{p}{(}\PY{n}{x}\PY{p}{[}\PY{l+m+mi}{0}\PY{p}{]}\PY{p}{)}\PY{o}{/}\PY{n}{df}\PY{p}{(}\PY{n}{x}\PY{p}{[}\PY{l+m+mi}{0}\PY{p}{]}\PY{p}{)}
\PY{n}{i} \PY{o}{=} \PY{l+m+mi}{0}
\PY{k}{while} \PY{n+nb}{abs}\PY{p}{(}\PY{n}{x}\PY{p}{[}\PY{n}{i}\PY{p}{]} \PY{o}{\PYZhy{}} \PY{n}{gx}\PY{p}{)} \PY{o}{\PYZgt{}} \PY{l+m+mf}{1e\PYZhy{}6}\PY{p}{:}
\PY{n}{gx} \PY{o}{=} \PY{n}{x}\PY{p}{[}\PY{n}{i}\PY{p}{]}
\PY{n}{x}\PY{p}{[}\PY{n}{i}\PY{o}{+}\PY{l+m+mi}{1}\PY{p}{]} \PY{o}{=} \PY{n}{x}\PY{p}{[}\PY{n}{i}\PY{p}{]}\PY{o}{\PYZhy{}} \PY{n}{f}\PY{p}{(}\PY{n}{x}\PY{p}{[}\PY{n}{i}\PY{p}{]}\PY{p}{)}\PY{o}{/}\PY{n}{df}\PY{p}{(}\PY{n}{x}\PY{p}{[}\PY{n}{i}\PY{p}{]}\PY{p}{)}
\PY{n}{i} \PY{o}{=} \PY{n}{i} \PY{o}{+} \PY{l+m+mi}{1}

\PY{n+nb}{print}\PY{p}{(}\PY{l+s+s1}{\PYZsq{}}\PY{l+s+s1}{Iterations until convergence: }\PY{l+s+s1}{\PYZsq{}}\PY{p}{,}\PY{n}{i}\PY{o}{+}\PY{l+m+mi}{1}\PY{p}{)}    

\PY{c+c1}{\PYZsh{}Part iii}
\PY{n+nb}{print}\PY{p}{(}\PY{l+s+s2}{\PYZdq{}}\PY{l+s+s2}{\PYZhy{}\PYZhy{}\PYZhy{}Part(iii) Plot\PYZhy{}\PYZhy{}\PYZhy{}}\PY{l+s+s2}{\PYZdq{}}\PY{p}{)}

\PY{c+c1}{\PYZsh{} plot the function}
\PY{n}{t} \PY{o}{=} \PY{n}{arange}\PY{p}{(}\PY{o}{\PYZhy{}}\PY{l+m+mi}{1}\PY{p}{,} \PY{l+m+mi}{2}\PY{p}{,} \PY{l+m+mf}{0.1}\PY{p}{)}
\PY{n}{yt} \PY{o}{=} \PY{n}{f}\PY{p}{(}\PY{n}{t}\PY{p}{)}

\PY{n}{pyplot}\PY{o}{.}\PY{n}{title}\PY{p}{(}\PY{p}{[}\PY{l+s+s1}{\PYZsq{}}\PY{l+s+s1}{Problem 1}\PY{l+s+s1}{\PYZsq{}}\PY{p}{]}\PY{p}{)}
\PY{n}{pyplot}\PY{o}{.}\PY{n}{subplot}\PY{p}{(}\PY{l+m+mi}{2}\PY{p}{,}\PY{l+m+mi}{1}\PY{p}{,}\PY{l+m+mi}{1}\PY{p}{)}
\PY{n}{pyplot}\PY{o}{.}\PY{n}{plot}\PY{p}{(}\PY{n}{t}\PY{p}{,} \PY{n}{yt}\PY{p}{,} \PY{l+s+s1}{\PYZsq{}}\PY{l+s+s1}{go}\PY{l+s+s1}{\PYZsq{}}\PY{p}{)}
\PY{n}{pyplot}\PY{o}{.}\PY{n}{plot}\PY{p}{(}\PY{p}{(}\PY{o}{\PYZhy{}}\PY{l+m+mi}{1}\PY{p}{,} \PY{l+m+mi}{1}\PY{p}{)}\PY{p}{,} \PY{p}{(}\PY{l+m+mi}{0}\PY{p}{,} \PY{l+m+mi}{0}\PY{p}{)}\PY{p}{,} \PY{l+s+s1}{\PYZsq{}}\PY{l+s+s1}{r}\PY{l+s+s1}{\PYZsq{}}\PY{p}{)}
\PY{n}{pyplot}\PY{o}{.}\PY{n}{grid}\PY{p}{(}\PY{k+kc}{True}\PY{p}{)}
\PY{n}{pyplot}\PY{o}{.}\PY{n}{axis}\PY{p}{(}\PY{p}{[}\PY{o}{\PYZhy{}}\PY{l+m+mi}{1}\PY{p}{,} \PY{l+m+mi}{1}\PY{p}{,} \PY{o}{\PYZhy{}}\PY{o}{.}\PY{l+m+mi}{5}\PY{p}{,} \PY{o}{.}\PY{l+m+mi}{5}\PY{p}{]}\PY{p}{)}
\PY{n}{pyplot}\PY{o}{.}\PY{n}{legend}\PY{p}{(}\PY{p}{[}\PY{l+s+s1}{\PYZsq{}}\PY{l+s+s1}{0.9*sin(x)\PYZhy{}x}\PY{l+s+s1}{\PYZsq{}}\PY{p}{,}\PY{l+s+s1}{\PYZsq{}}\PY{l+s+s1}{f(x)=0}\PY{l+s+s1}{\PYZsq{}}\PY{p}{]}\PY{p}{)}
\PY{n}{pyplot}\PY{o}{.}\PY{n}{subplot}\PY{p}{(}\PY{l+m+mi}{2}\PY{p}{,}\PY{l+m+mi}{1}\PY{p}{,}\PY{l+m+mi}{2}\PY{p}{)}
\PY{n}{pyplot}\PY{o}{.}\PY{n}{plot}\PY{p}{(}\PY{n+nb}{range}\PY{p}{(}\PY{l+m+mi}{0}\PY{p}{,}\PY{n}{i}\PY{o}{+}\PY{l+m+mi}{1}\PY{p}{)}\PY{p}{,} \PY{n}{x}\PY{p}{[}\PY{l+m+mi}{0}\PY{p}{:}\PY{n}{i}\PY{o}{+}\PY{l+m+mi}{1}\PY{p}{]}\PY{p}{,} \PY{l+s+s1}{\PYZsq{}}\PY{l+s+s1}{b\PYZhy{}}\PY{l+s+s1}{\PYZsq{}}\PY{p}{)}
\PY{n}{pyplot}\PY{o}{.}\PY{n}{plot}\PY{p}{(}\PY{n+nb}{range}\PY{p}{(}\PY{l+m+mi}{0}\PY{p}{,}\PY{n}{ia}\PY{p}{)}\PY{p}{,} \PY{n}{xa}\PY{p}{[}\PY{l+m+mi}{0}\PY{p}{:}\PY{n}{ia}\PY{p}{]}\PY{p}{,} \PY{l+s+s1}{\PYZsq{}}\PY{l+s+s1}{c\PYZhy{}}\PY{l+s+s1}{\PYZsq{}}\PY{p}{)}
\PY{n}{pyplot}\PY{o}{.}\PY{n}{grid}\PY{p}{(}\PY{k+kc}{True}\PY{p}{)}
\PY{n}{pyplot}\PY{o}{.}\PY{n}{axis}\PY{p}{(}\PY{p}{[}\PY{l+m+mi}{0}\PY{p}{,} \PY{l+m+mi}{100}\PY{p}{,} \PY{o}{\PYZhy{}}\PY{o}{.}\PY{l+m+mi}{25}\PY{p}{,} \PY{l+m+mi}{1}\PY{p}{]}\PY{p}{)}
\PY{n}{pyplot}\PY{o}{.}\PY{n}{legend}\PY{p}{(}\PY{p}{[}\PY{l+s+s1}{\PYZsq{}}\PY{l+s+s1}{Newton\PYZhy{}Raphson}\PY{l+s+s1}{\PYZsq{}}\PY{p}{,}\PY{l+s+s1}{\PYZsq{}}\PY{l+s+s1}{Iteration}\PY{l+s+s1}{\PYZsq{}}\PY{p}{]}\PY{p}{)}
\PY{n}{pyplot}\PY{o}{.}\PY{n}{show}\PY{p}{(}\PY{p}{)}

\PY{c+c1}{\PYZsh{}Problem 5}
\PY{n+nb}{print}\PY{p}{(}\PY{l+s+s2}{\PYZdq{}}\PY{l+s+s2}{\PYZhy{}\PYZhy{}\PYZhy{}\PYZhy{}\PYZhy{}\PYZhy{}\PYZhy{}Problem 5\PYZhy{}\PYZhy{}\PYZhy{}\PYZhy{}\PYZhy{}\PYZhy{}\PYZhy{}}\PY{l+s+s2}{\PYZdq{}}\PY{p}{)}

\PY{c+c1}{\PYZsh{}Define the function}
\PY{n}{f2} \PY{o}{=} \PY{k}{lambda} \PY{n}{x}\PY{p}{:} \PY{n}{x}\PY{o}{*}\PY{o}{*}\PY{l+m+mi}{5} \PY{o}{\PYZhy{}} \PY{l+m+mf}{4.5}\PY{o}{*}\PY{n}{x}\PY{o}{*}\PY{o}{*}\PY{l+m+mi}{4} \PY{o}{+}\PY{l+m+mf}{4.55}\PY{o}{*}\PY{n}{x}\PY{o}{*}\PY{o}{*}\PY{l+m+mi}{3} \PY{o}{+} \PY{l+m+mf}{2.675}\PY{o}{*}\PY{n}{x}\PY{o}{*}\PY{o}{*}\PY{l+m+mi}{2} \PY{o}{\PYZhy{}}\PY{l+m+mf}{3.3}\PY{o}{*}\PY{n}{x} \PY{o}{\PYZhy{}}\PY{l+m+mf}{1.4375}
\PY{n}{df2} \PY{o}{=} \PY{k}{lambda} \PY{n}{x}\PY{p}{:} \PY{l+m+mi}{5}\PY{o}{*}\PY{n}{x}\PY{o}{*}\PY{o}{*}\PY{l+m+mi}{4} \PY{o}{\PYZhy{}}\PY{l+m+mi}{4}\PY{o}{*}\PY{l+m+mf}{4.5}\PY{o}{*}\PY{n}{x}\PY{o}{*}\PY{o}{*}\PY{l+m+mi}{3} \PY{o}{+}\PY{l+m+mi}{3}\PY{o}{*}\PY{l+m+mf}{4.55}\PY{o}{*}\PY{n}{x}\PY{o}{*}\PY{o}{*}\PY{l+m+mi}{2} \PY{o}{+} \PY{l+m+mi}{2}\PY{o}{*}\PY{l+m+mf}{2.675}\PY{o}{*}\PY{n}{x} \PY{o}{\PYZhy{}}\PY{l+m+mf}{3.3}

\PY{c+c1}{\PYZsh{}Begin Algo}
\PY{n}{x0} \PY{o}{=} \PY{o}{\PYZhy{}}\PY{l+m+mf}{0.6}\PY{p}{;}
\PY{n}{x} \PY{o}{=} \PY{n}{zeros}\PY{p}{(}\PY{l+m+mi}{100}\PY{p}{)}
\PY{n}{x}\PY{p}{[}\PY{l+m+mi}{0}\PY{p}{]}\PY{o}{=}  \PY{n}{x0}
\PY{n}{gx} \PY{o}{=} \PY{n}{x}\PY{p}{[}\PY{l+m+mi}{0}\PY{p}{]} \PY{o}{\PYZhy{}} \PY{n}{f2}\PY{p}{(}\PY{n}{x}\PY{p}{[}\PY{l+m+mi}{0}\PY{p}{]}\PY{p}{)}\PY{o}{/}\PY{n}{df2}\PY{p}{(}\PY{n}{x}\PY{p}{[}\PY{l+m+mi}{0}\PY{p}{]}\PY{p}{)}
\PY{n}{i} \PY{o}{=} \PY{l+m+mi}{0}
\PY{k}{while} \PY{n+nb}{abs}\PY{p}{(}\PY{n}{x}\PY{p}{[}\PY{n}{i}\PY{p}{]} \PY{o}{\PYZhy{}} \PY{n}{gx}\PY{p}{)} \PY{o}{\PYZgt{}} \PY{l+m+mf}{1e\PYZhy{}6}\PY{p}{:}
\PY{n}{gx} \PY{o}{=} \PY{n}{x}\PY{p}{[}\PY{n}{i}\PY{p}{]}
\PY{n}{x}\PY{p}{[}\PY{n}{i}\PY{o}{+}\PY{l+m+mi}{1}\PY{p}{]} \PY{o}{=} \PY{n}{x}\PY{p}{[}\PY{n}{i}\PY{p}{]}\PY{o}{\PYZhy{}} \PY{n}{f2}\PY{p}{(}\PY{n}{x}\PY{p}{[}\PY{n}{i}\PY{p}{]}\PY{p}{)}\PY{o}{/}\PY{n}{df2}\PY{p}{(}\PY{n}{x}\PY{p}{[}\PY{n}{i}\PY{p}{]}\PY{p}{)}
\PY{n}{i} \PY{o}{=} \PY{n}{i} \PY{o}{+} \PY{l+m+mi}{1}

\PY{n+nb}{print}\PY{p}{(}\PY{l+s+s1}{\PYZsq{}}\PY{l+s+s1}{Iterations until convergence: }\PY{l+s+s1}{\PYZsq{}}\PY{p}{,}\PY{n}{i}\PY{o}{+}\PY{l+m+mi}{1}\PY{p}{)}

\PY{c+c1}{\PYZsh{}M\PYZhy{}K}
\PY{n}{E} \PY{o}{=} \PY{n}{zeros}\PY{p}{(}\PY{n}{i}\PY{o}{+}\PY{l+m+mi}{1}\PY{p}{)}
\PY{k}{for} \PY{n}{j} \PY{o+ow}{in} \PY{n+nb}{range}\PY{p}{(}\PY{n}{i}\PY{o}{+}\PY{l+m+mi}{1}\PY{p}{)}\PY{p}{:}
\PY{n}{E}\PY{p}{[}\PY{n}{j}\PY{p}{]}\PY{o}{=}\PY{n+nb}{abs}\PY{p}{(}\PY{n}{x}\PY{p}{[}\PY{n}{j}\PY{p}{]}\PY{o}{\PYZhy{}}\PY{n}{x}\PY{p}{[}\PY{n}{j}\PY{o}{+}\PY{l+m+mi}{1}\PY{p}{]}\PY{p}{)}

\PY{n}{K} \PY{o}{=} \PY{n}{zeros}\PY{p}{(}\PY{n}{i}\PY{o}{\PYZhy{}}\PY{l+m+mi}{1}\PY{p}{)}
\PY{k}{for} \PY{n}{j} \PY{o+ow}{in} \PY{n+nb}{range}\PY{p}{(}\PY{n}{i}\PY{o}{\PYZhy{}}\PY{l+m+mi}{1}\PY{p}{)}\PY{p}{:}
\PY{n}{K}\PY{p}{[}\PY{n}{j}\PY{p}{]}\PY{o}{=}\PY{p}{(}\PY{n}{E}\PY{p}{[}\PY{n}{j}\PY{o}{+}\PY{l+m+mi}{1}\PY{p}{]}\PY{o}{/}\PY{n}{E}\PY{p}{[}\PY{n}{j}\PY{p}{]}\PY{p}{)}

\PY{n}{Kmin}\PY{o}{=}\PY{n+nb}{min}\PY{p}{(}\PY{n}{K}\PY{p}{)}
\PY{n+nb}{print}\PY{p}{(}\PY{l+s+s1}{\PYZsq{}}\PY{l+s+s1}{K \PYZti{} }\PY{l+s+s1}{\PYZsq{}}\PY{p}{,} \PY{n}{Kmin}\PY{p}{)}
\PY{n}{M}\PY{o}{=}\PY{l+m+mi}{1}\PY{o}{/}\PY{p}{(}\PY{l+m+mi}{1}\PY{o}{\PYZhy{}}\PY{n}{Kmin}\PY{p}{)}
\PY{n+nb}{print}\PY{p}{(}\PY{l+s+s1}{\PYZsq{}}\PY{l+s+s1}{M \PYZti{} }\PY{l+s+s1}{\PYZsq{}}\PY{p}{,} \PY{n}{M}\PY{p}{)}

\PY{c+c1}{\PYZsh{}Plot}
\PY{c+c1}{\PYZsh{} plot the function}
\PY{n}{t} \PY{o}{=} \PY{n}{arange}\PY{p}{(}\PY{o}{\PYZhy{}}\PY{l+m+mi}{1}\PY{p}{,} \PY{l+m+mi}{5}\PY{p}{,} \PY{l+m+mf}{0.1}\PY{p}{)}
\PY{n}{y2t} \PY{o}{=} \PY{n}{f2}\PY{p}{(}\PY{n}{t}\PY{p}{)}

\PY{n}{pyplot}\PY{o}{.}\PY{n}{title}\PY{p}{(}\PY{l+s+s1}{\PYZsq{}}\PY{l+s+s1}{Problem 2}\PY{l+s+s1}{\PYZsq{}}\PY{p}{)}
\PY{n}{pyplot}\PY{o}{.}\PY{n}{plot}\PY{p}{(}\PY{n}{t}\PY{p}{,} \PY{n}{y2t}\PY{p}{,} \PY{l+s+s1}{\PYZsq{}}\PY{l+s+s1}{g\PYZhy{}}\PY{l+s+s1}{\PYZsq{}}\PY{p}{)}
\PY{n}{pyplot}\PY{o}{.}\PY{n}{plot}\PY{p}{(}\PY{p}{(}\PY{o}{\PYZhy{}}\PY{l+m+mi}{5}\PY{p}{,} \PY{l+m+mi}{5}\PY{p}{)}\PY{p}{,} \PY{p}{(}\PY{l+m+mi}{0}\PY{p}{,} \PY{l+m+mi}{0}\PY{p}{)}\PY{p}{,} \PY{l+s+s1}{\PYZsq{}}\PY{l+s+s1}{r}\PY{l+s+s1}{\PYZsq{}}\PY{p}{)}
\PY{n}{pyplot}\PY{o}{.}\PY{n}{grid}\PY{p}{(}\PY{k+kc}{True}\PY{p}{)}
\PY{n}{pyplot}\PY{o}{.}\PY{n}{axis}\PY{p}{(}\PY{p}{[}\PY{o}{\PYZhy{}}\PY{l+m+mi}{2}\PY{p}{,} \PY{l+m+mi}{2}\PY{p}{,} \PY{o}{\PYZhy{}}\PY{l+m+mi}{1}\PY{p}{,} \PY{l+m+mi}{1}\PY{p}{]}\PY{p}{)}
\PY{n}{pyplot}\PY{o}{.}\PY{n}{legend}\PY{p}{(}\PY{p}{[}\PY{l+s+s1}{\PYZsq{}}\PY{l+s+s1}{5*x\PYZca{}4 \PYZhy{}4*4.5*x\PYZca{}3 +3*4.55*x\PYZca{}2 + 2*2.675*x \PYZhy{}3.3}\PY{l+s+s1}{\PYZsq{}}\PY{p}{,}\PY{l+s+s1}{\PYZsq{}}\PY{l+s+s1}{f(x)=0}\PY{l+s+s1}{\PYZsq{}}\PY{p}{]}\PY{p}{)}
\PY{n}{pyplot}\PY{o}{.}\PY{n}{show}\PY{p}{(}\PY{p}{)}
\end{Verbatim}

\begin{Verbatim}[commandchars=\\\{\}]
-------Problem 4-------
---Part(i) Simple Iterations---
Iterations until convergence:  106
---Part(ii) Newton-Raphson---
Iterations until convergence:  6
---Part(iii) Plot---

\end{Verbatim}

\begin{center}
	\adjustimage{max size={0.9\linewidth}{0.9\paperheight}}{output_0_1.png}
\end{center}
{ \hspace*{\fill} \\}

\begin{Verbatim}[commandchars=\\\{\}]
-------Problem 5-------
Iterations until convergence:  18
K \textasciitilde{}  0.5000022589142745
M \textasciitilde{}  2.0000090356979197

\end{Verbatim}

\begin{center}
	\adjustimage{max size={0.9\linewidth}{0.9\paperheight}}{output_0_3.png}
\end{center}
{ \hspace*{\fill} \\}

\begin{Verbatim}[commandchars=\\\{\}]
{\color{incolor}In [{\color{incolor} }]:} 
\end{Verbatim}
	
\end{document}
